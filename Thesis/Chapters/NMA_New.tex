\chapter{Network Meta Analysis}

\section{Building a Network of Evidence}

Consider a set of $N$ two-arm randomised-controlled trials (RCTs). In each trial $i \in 1,\ldots,N$, the patients are randomised to recieve a treatment $A_i$, or a placebo $P_i$. This can be represented as N graphs with two nodes, $A_i$ and $P_i$, connected by an edge representing the trial comparing $A_i$ and $P_i$. It is useful at this stage to recall the formal definition of an (undirected) graph. 

\begin{definition}{Graph}
A  A \textbf{Graph} is an ordered triple $G = (V, E, \varphi)$. Where  $V$ is a set of nodes, E is a set  of edges, and $\varphi : E \to \{\{ x, y \} | x, y \in V \ \text{such that} \ x \neq y \}$ is an  \textbf{incidence function} mapping every edge to a pair of verticesa.
\end{definition}

We can construct $N$ graphs under the formal definition. Namely, for trial $T_i$, we have $G_i = (V_i, E_i, \varphi_i)$ where $V_i = \{ A_i, P_i \}$, $E_i = \{T_i\}$ and $\varphi_i : E_i \to \{ \{ x, y \} | x, y \in V_i \ \text{such that} \  x \neq y \}$. For construction of the grpahs, we can drop the subscript on $P_i$, and take the placebo as a reference treatment. This is done under the assumption that the effect of placebo is constant across all trials. This is a strong assumption, and implications of this are discussed later. Under this assumption however, each $V_i$ now contains a common element, $P$.

Let

\begin{align*}
    V_{trts} &= \bigcup_{i=1}^{N} V_i \\
    E_{trials} &= \bigcup_{i=1}^{N} E_i
.\end{align*}

The incidence function becomes 

\[
    \varphi : E_{trials} \to \{ \{x, y \} | x, y \in V_{trts} \ \text{such that} \ x = P \}
.\] 

Then the ordered triple $G = (V_{trts}, E_{trials}, \varphi)$ is the network of evidence given by these two arm trials that forms the basis of a network meta analysis. This process expands to trials that compare more than two treatments by weighting the edges by the number of trials making that particular comparison.

\begin{figure}[h]
    \centering
    \includesvg{../figures/combined_evidence.svg}
    \caption{Visualisation of combining trials into a network of evidence}
    \label{fig:comb_evi}
\end{figure}

\section{Deriving the NMA Model}

The derivation in this chapter is based on the work of~\ref{phillippo2024}, but modified for this project. Under an NMA framework, we have $J$ RCTs investigating a subset $K_j$ ($j = 1, \ldots, J$) of $K$ treatments. In this project, $|K_j| = 2 \ \forall \ j$. Depending on data avialability, we may have IPD for some studies, and only aggregate data for the remaining. 

\begin{definition}{General IPD Meta-Regression Model}
    LLet $y_{ijk}$ be the IPD outcome for individual $i = 1, \ldots, N_{kj}$ in study $j$ recieving treatment $k \in K_j$ given the likelihood distribution $\pi_{Ind}(\theta_{ijk})$. 
    \begin{align*}
        y_{ijk} &\sim \pi_{Ind}(\theta_{ijk}) \\
        g(\theta_{ijk}) &= \mu_j + \bf{x}^T_{ijk}(\bf{\beta}_1 + \bf{\beta}_{2,k}) + \gamma_k \\
                        &= \eta_{jk}(\bf{x}_{ijk})
    \end{align*}
    Here, $g$ links the likelihood parameter $\theta_{ijk}$ to $\eta_{jk}(\bf{x}_{ijk})$. The $\mu_j$ are study-specific intercepts, and $\bf{\beta}_1$, $\bf{\beta}_{2,k}$ are regression coefficients for prognostic and effect-modifying covariates respectively. Additionally, the $\gamma_k$ are individual-level treatment effects. For the reference treatment, $\bf{\beta}_{2,1} = \gamma_1 = 0$.
    \label{def:mlnmrIPD}
\end{definition}

\section{Survival ML-NMR}
Each study reports a pair $y_{ijk} = \{t_{ijk}, c_{ijk}\}$ consisting of outcome times $t_{ijk}$ and censoring indicators $c_{ijk}$, either from IPD or reconstructed IPD. For IPD studies, the covariates $\bf{x}_{ijk}$ will naturaly be avialble, but for aggregate studies (those for which pseudo-IPD has been re-created), only the joint covariate distribution of covariates at baseline, denoted $f_{jk}$. \\

Let $S_{jk}(t|\bf{x})$ and $h_{jk}(t|\bf{x})$ be the survival and hazard functions conditional on the covariates $\bf{x}$. Then the individual conditional likelohood contributions for each time $t_{ijk}$ in the IPD studies are given by

\begin{equation}
    L_{ijk|x}^{Con}(\zeta;t_{ijk},c_{ijk},\bf{x}_{ijk}) = S_{jk}(t_{ijk}|\bf{x}_{ijk})h_{jk}(t_{ijk}|\bf{x}_{ijk})^{c_{ijk}}
\end{equation}

\bf{The forms of $S$ and $h$ depend on the specific models chosen}.