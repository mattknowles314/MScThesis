\chapter{Network Meta Analysis Theory}

\section{Building a Network of Evidence}

Consider a set of $N$ two-arm randomised-controlled trials (RCTs). In each trial $i \in 1,\ldots,N$, the patients are randomised to recieve a treatment $A_i$, or a placebo $P_i$. This can be represented as N graphs with two nodes, $A_i$ and $P_i$, connected by an edge representing the trial comparing $A_i$ and $P_i$. It is useful at this stage to recall the formal definition of an (undirected) graph. 

\begin{definition}{Graph}
A  A \textbf{Graph} is an ordered triple $G = (V, E, \varphi)$. Where  $V$ is a set of nodes, E is a set  of edges, and $\varphi : E \to \{\{ x, y \} | x, y \in V \ \text{such that} \ x \neq y \}$ is an  \textbf{incidence function} mapping every edge to a pair of verticesa.
\end{definition}

We can construct $N$ graphs under the formal definition. Namely, for trial $T_i$, we have $G_i = (V_i, E_i, \varphi_i)$ where $V_i = \{ A_i, P_i \}$, $E_i = \{T_i\}$ and $\varphi_i : E_i \to \{ \{ x, y \} | x, y \in V_i \ \text{such that} \  x \neq y \}$. For construction of the grpahs, we can drop the subscript on $P_i$, and take the placebo as a reference treatment. This is done under the assumption that the effect of placebo is constant across all trials. This is a strong assumption, and implications of this are discussed later. Under this assumption however, each $V_i$ now contains a common element, $P$.

Let

\begin{align*}
    V_{trts} &= \bigcup_{i=1}^{N} V_i \\
    E_{trials} &= \bigcup_{i=1}^{N} E_i
.\end{align*}

The incidence function becomes 

\[
    \varphi : E_{trials} \to \{ \{x, y \} | x, y \in V_{trts} \ \text{such that} \ x = P \}
.\] 

Then the ordered triple $G = (V_{trts}, E_{trials}, \varphi)$ is the network of evidence given by these two arm trials that forms the basis of a network meta analysis. This process expands to trials that compare more than two treatments by weighting the edges by the number of trials making that particular comparison.

\begin{figure}[h]
    \centering
    \includesvg{../figures/combined_evidence.svg}
    \caption{Visualisation of combining trials into a network of evidence}
    \label{fig:comb_evi}
\end{figure}

\section{Standard NMA Model}
Let $d_{ab}$ denote the relative effect of treatment $b$ versus treatment $a$. Suppose we have summary outcomes $y_{jk}$ of treatment $k$ in study $j$, the standard NMA model is written as in Equation~\ref{eq:standardNMA1}-Equation~\ref{eq:standardNMA2}. This summary outcome may be, for example, HRs, or RMST values. In Equation~\ref{eq:standardNMA1}, $\pi_{Agg}$ is a suitable likelihood for the aggregate data, and $\theta_{jk}$ represents the expected summary outcome of treatment $k$ in study $j$. The link function $g$ serves to transform $\theta_{jk}$ onto the linear predictor scale. In Equation~\ref{eq:standardNMA2}, $\mu_j$ and $\delta_{jk}$ are study-specific intercepts and study-specific relative effect of treatment $k$ versus the reference treatment.

\begin{align}
    y_{jk} &\sim \pi_{Agg}(\theta_{jk}) \label{eq:standardNMA1} \\
    g(\theta_{jk}) &= \mu_j + \delta_{jk} \label{eq:standardNMA2}
\end{align}

The two types of NMA are fixed effect (FE) and relative effects (RE) NMAs. In an FE NMA, $\delta_{jk} = d_{1k} = d_k$, with $d_1 = 0$. In an RE NMA, $\delta_{jk} \sim N(d_k, \tau^2)$ for the heterogeneity variance $\tau^2$, with $\delta_j1 = d_1 = 0$. \\

The standard NMA model assumes that any effect modifiers, i.e covariates that alter the relative effect on a given scale of an active treatment versus control, are balanced across populations. While this can often be a valid assumption, methods such as Matching-Adjusted-Indirect-Comparisons (MAICs), Simulated Treatment Comparisons (STCs), and Multi-Level Network-Meta-Regression (ML-NMR) have saught to relax this assumption by using IPD from at least one of the studies in a population.

\section{Multi-Level Network Meta Regression}
The derivation in this chapter is based on the work of~\cite{phillippo2024}. Under an NMA framewrok, there are $J$ RCTs investigating a subset $K_j \subset K$ $(j = 1,\ldots,J)$ treatments. In this project, $|K_j| = 2 \ \forall \ j$. Depending on data avialbility, we may have individual-patient-data (IPD) for some studies, and only aggregate data for the remaining. This would be an ideal scenario, however it is not always the case.

\begin{definition}{General IPD Meta-Regression Model}
    LLet $y_{ijk}$ be the IPD outcome for individual $i = 1, \ldots, N_{kj}$ in study $j$ recieving treatment $k \in K_j$ given the likelihood distribution $\pi_{Ind}(\theta_{ijk})$. 
    \begin{align*}
        y_{ijk} &\sim \pi_{Ind}(\theta_{ijk}) \\
        g(\theta_{ijk}) &= \mu_j + x^T_{ijk}(\beta_1 + \beta_{2,k}) + \gamma_k \\
                        &= \eta_{jk}(x_{ijk})
    \end{align*}
    Here, $g$ links the likelihood parameter $\theta_{ijk}$ to $\eta_{jk}(x_ijk)$. The $\mu_j$ are study-specific intercepts, and $\beta_1$, $\beta_{2,k}$ are regression coefficients for prognostic and effect-modifying covariates respectively. Additionally, the $\gamma_k$ are individual-level treatment effects. For the reference treatment, $\beta_{2,1} = \gamma_1 = 0$.
    \label{def:mlnmrIPD}
\end{definition}

It is clear to see how the model in Definition~\ref{def:mlnmrIPD} extends Equations~\ref{eq:standardNMA1}-\ref{eq:standardNMA2}. Let $\xi = \{\mu_j, \beta_1, \beta_{2,k}, \gamma_k | \forall \ j,k \}$ be the parameter space. Using $\xi$, we can denote the individual conditional likelihood function by $L_{ijk|x}^{\text{Con}}(\xi; y_{ijk}, x_{ijk})$. The form of $L_{ijk|x}^{\text{Con}}$ depends on $\pi_{Ind}$, $g$, and $\eta_{jk}$. 

By integrating the individual conditional likelihood ovr the joint covariate distributoin $f_{jk}$, we obtain Equation~\ref{eq:marg}, the individual marginal likelihood function.

\begin{equation}
    L_{ijk}^{\text{Mar}}(\xi; y_{ijk}) = \int_{\mathfrak{X}} L_{ijk|x}^{\text{Con}}(\xi; y_{ijk}, x)f_{jk}(x)dx
    \label{eq:marg}
\end{equation}

It is clear from Equation~\ref{eq:marg} that $L_{ijk}^{\text{Mar}}$ does not depend on x. Let $i$ be an individual, on treatment $k$ in study $j$ with outcome $y_{ijk}$. If we don't know the covariate vector for $i$, $x_{ijk}$, but we do know $f_{jk}$, then we know that the likelihood contribution of $i$ is given by Equation~\ref{eq:marg}. \\

It is likeliy that a closed-form of Equation~\ref{eq:marg} does not exist. We can therefore take a set of $N$ integration points, $\hat{x}$ from $f_{jk}$, giving Equation~\ref{eq:margApprox}.

\begin{equation}
    L_{ijk}^{\text{Mar}}(\xi; y_{ijk}) \approx \frac{1}{N}\sum_{\hat{x}}L_{ijk|x}^{\text{Con}}(\xi;y_{ijk},x)
    \label{eq:margApprox}
\end{equation}

Consider a summary outcome $y_{\hat{jk}}$ aggregated over all individuals on treatment $k$ in study $j$. Each individual $i$ in on treatment $k$ in study $i$ contributes to the aggregate likelihood. Let $y_{ijk}$ denote the observed value of this summary measure for individual $i$. The aggregate maginal likelihood function is then the product of these $y_ijk$ up to a normalising constant, as in Equation~\ref{aggMargLik}.

\begin{equation}
    L_{\hat{jk}}^{\text{Mar}} \propto \prod_{i = 1}^{N_{jk}}L_{ijk}^{\text{Mar}}(\xi; y_{ijk})
    \label{aggMargLik}
\end{equation}

The full, general, ML-NMR model is then given by 

\begin{definition}{General ML-NMR Model}
    IIndividual:
    \begin{align}
        L_{ijk|x}^{\text{Con}}(\xi;y_{ijk},x_{ijk}) &= \pi_{\text{Ind}}(y_{ijk}|\theta_{ijk}) \\
        g(\theta_{ijk}) &= \eta_{jk}(x_{ijk}) = \mu_j + x_{ijk}^T(\beta_1 + \beta_{2,k}) + \gamma_k
    \end{align}
    Aggregate:
    \begin{align}
        L_{ijk}^{\text{Mar}}(\xi; y_{ijk}) &= \int_{\mathfrak{X}} L_{ijk|x}^{\text{Con}}(\xi; y_{ijk}, x)f_{jk}(x)dx \\
        L_{\hat{jk}}^{\text{Mar}} &\propto \prod_{i = 1}^{N_{jk}}L_{ijk}^{\text{Mar}}(\xi; y_{ijk})
    \end{align}
        
\end{definition}

Under a Bayesian framework, priors are place on $\mu_j$, $\beta_1$, $\beta_{2,l}$, and $\gamma_k$.

\section{Survival ML-NMR}
Each study reports a pair $y_{ijk} = \{t_{ijk}, c_{ijk}\}$ consisting of outcome times $t_{ijk}$ and censoring indicators $c_{ijk}$, either from IPD or reconstructed IPD. For IPD studies, the covariates $x_{ijk}$ will naturaly be avialble, but for aggregate studies (those for which pseudo-IPD has been re-created), only the joint covariate distribution of covariates at baseline, denoted $f_{jk}$. \\
The censoring indicator is defined as in Equation~\ref{censInd}. 

\begin{equation}
    c_{ijk} = \begin{cases}
        1 \ \text{If inividual has an event} \\
        0 \ \text{If individual is censored}
    \end{cases}
    \label{censInd}  
\end{equation}

In practice, the censoring indicator can be the other way round. Therefore, when cleaning data for this dissertation, manual reversing of the censoring indicator was conducted to ensure all data used the same definition.

Let $S_{jk}(t|\bf{x})$ and $h_{jk}(t|x)$ be the survival and hazard functions conditional on the covariates $x$. Then the individual conditional likelihood contributions for each time $t_{ijk}$ in the IPD studies are given by

\begin{equation}
    L_{ijk|x}^{Con}(\zeta;t_{ijk},c_{ijk},x_{ijk}) = S_{jk}(t_{ijk}| x_{ijk})h_{jk}(t_{ijk}|x_{ijk})^{c_{ijk}}
\end{equation}

The forms of $S$ and $h$ depend on the specific survival models chosen. Starting from Equation~\ref{eq:marg}, the marginal likelihood equations for each event/censoring time in the aggregate data studies can be derived. Substituting $y_{ijk} = \{t_{ijk}, c_{ijk}\}$

\begin{align}
    L_{ijk}^{\text{Mar}}(\xi; y_{ijk}) &= \int_{\mathfrak{X}} L_{ijk|x}^{\text{Con}}(\xi; y_{ijk}, x)f_{jk}(x)dx \\ 
    L_{ijk}^{\text{Mar}}(\xi; t_{ijk}, c_{ijk}) &= \int_{\mathfrak{X}} L_{ijk|x}^{\text{Con}}(\xi; t_{ijk}, c_{ijk}, x)f_jk(x)dx \\
                                       &= \int_{\mathfrak{X}} S_{jk}(t_{ijk}| x_{ijk})h_{jk}(t_{ijk}|x_{ijk})^{c_{ijk}} f_{jk}(x)dx\label{eq:survint}
\end{align}

As with Equation~\ref{eq:margApprox}, Equation~\ref{eq:survint} can be evalueated with quasi-Monte Carlo integration to obtain Equation

\begin{equation}
    L_{ijk}^{\text{Mar}}(\xi; t_{ijk}, c_{ijk}) = \frac{1}{N}\sum_{\hat{x}}S_{jk}(t_{ijk}|\hat{x})h_{jk}(t_{ijk}|\hat{x})^{c_{ijk}}
\end{equation}

All of these calculations are performed in Stan using the \verb|multinma| R package~\cite{multinma}.

\section{Population-average estimates}
Recall $d_{ab}$ is the relative effect of treatment b versus treatment a. Let $d_{ab(P)}$ be the population-average relative effect of b versus a in population P. $d_{ab(P)}$ can be calculated as in Equation~\ref{popavd}. 

\begin{align}
    d_{ab(P)} &= \int_{\mathfrak{X}}(\eta_{(P)b}(x) - \eta_{(P)a}(x))f_{(P)}(x)dx \label{popavgdInt} \\
              &= \gamma_b - \gamma_a + \bar{x}^T_{(P)}(\beta_{2,b}-\beta_{2,a}) \label{popavd}
\end{align}

For this dissertation, The estimates considered for results were the RMST and median OS. The survival and hazard functions were also analysed but were not primary outcomes for the NMA.

\subsection{Survival function}
Let $\bar{S}_{(P)k}(t)$ be the population-average marginal survival probability of treatmentt $k$ in population $P$ at time $t$. $\bar{S}_{(P)k}(t)$ is obtained by integrating $S_{(P)k}(t|x)$ over $f_{(P)}(x)$, as in Equation~\ref{popavgS}.

\begin{equation}
    \bar{S}_{(P)k}(t) = \int_{\mathfrak{X}}S_{(P)k}(t|x)f_{(P)}(x)dx
    \label{popavgS}
\end{equation}

\subsection{Hazard function}

The population-average marginal hazard function and cumulative hazard function are given by Equation~\ref{popavghaz}, and Equation~\ref{popavgcumhaz}, respectively.

\begin{align}
    \bar{h}_{(P)k}(t) &= \frac{\int_{\mathfrak{X}}S_{(P)k}(t|x)h_{(P)k}(t|x)f_{(P)k}(x)}{\bar{S}_{(P)k}(t)} \label{popavghaz}\\
    \bar{H}{(P)k}(t) &= -\log(\bar{S}_{(P)k}(t)) \label{popavgcumhaz}
\end{align}

\subsection{RMST}
Let $x$ be some time horizon. The population-average marginal RMST follows from Definition~\ref{def:rmst}.

\begin{equation}
    RMST_{(P)k}(x) = \int^{x}_{0} \bar{S}_{(P)k}(t)dt.
\end{equation}

\subsection{Median OS}
In general, the $\alpha\%$ quantile is obtained by solving

\begin{equation}
    \bar{S}_{(P)k}(t) = 1 - \alpha.
\end{equation}

Since the median OS is a special case of this with $\alpha = \frac{1}{2}$, the population-average marginal median OS, $m$ is estimated by Equation~\ref{popavgmed}. 

\begin{equation}
    \bar{S}_{(P)k}(m) = \frac{1}{2}
    \label{popavgmed}
\end{equation}