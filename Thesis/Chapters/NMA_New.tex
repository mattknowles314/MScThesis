\chapter{Network Meta Analysis Theory}\label{nmatheory}

\section{Building a Network of Evidence}

Consider a set of $J$ two-arm randomised-controlled trials (RCTs). In each trial $j \in 1,\ldots,J$, the patients are randomised to recieve a treatment $A_j$, or a placebo $P_j$. This can be represented as $J$ graphs with two nodes, $A_j$ and $P_j$, connected by an edge representing the trial comparing $A_j$ and $P_j$. It is useful at this stage to recall the formal definition of an (undirected) graph, as presented in Defintion~\ref{graph}. 

\begin{definition}[label=graph]{Graph}
A  A \textbf{Graph} is an ordered triple $G = (V, E, \varphi)$. Where  $V$ is a set of nodes, E is a set  of edges, and $\varphi : E \to \{\{ x, y \} | x, y \in V \ \text{such that} \ x \neq y \}$ is an  \textbf{incidence function} mapping every edge to a pair of verticesa.
\end{definition}

We can construct $J$ graphs under the formal definition. Namely, for trial $T_i$, we have $G_i = (V_i, E_i, \varphi_i)$ where $V_i = \{ A_i, P_i \}$, $E_i = \{T_i\}$ and $\varphi_i : E_i \to \{ \{ x, y \} | x, y \in V_i \ \text{such that} \  x \neq y \}$. For construction of the grpahs, we drop the subscript on $P_i$, and take placebo (or any common treatment, not necessarily placebo) as a reference treatment. This is done under the assumption that the effect of placebo is constant across all trials. This is a strong assumption, and implications of this are discussed later. Under this assumption however, each $V_i$ now contains a common element, $P$.

Let

\begin{align}
    V_{trts} &= \bigcup_{j=1}^{J} V_j \\
    E_{trials} &= \bigcup_{j=1}^{J} E_j
.\end{align}

The incidence function is given in Equation~\ref{incidenceFunc}.

\begin{equation}
    \varphi : E_{trials} \to \{ \{x, y \} | x, y \in V_{trts} \ \text{such that} \ x = P \}
    \label{incidenceFunc}
\end{equation}

Then the ordered triple $G = (V_{trts}, E_{trials}, \varphi)$ is the network of evidence given by these two arm trials that forms the basis of an NMA. This process expands to trials that compare more than two treatments by weighting the edges by the number of trials making that particular comparison. Figure~\ref{fig:comb_evi} illustrates combining treatments into a network of evidence.\\

\begin{figure}[h]
    \centering
    \includesvg{../figures/combined_evidence.svg}
    \caption{Visualisation of combining trials into a network of evidence}
    \label{fig:comb_evi}
\end{figure}

In some scenarios, a head-to-head (HtH) clinical trial may be included in an NMA. Including a HtH trial involves special consideration, since these trials do not compare a treatment versus a placebo, but against another active treatment. From a graph-theoretic perspective, this can create a loop. A loop is an edge that connects a vertex to itself, and makes a graph non-simple.

\section{Standard NMA Model}
Let $d_{ab}$ denote the relative effect of treatment $b$ versus treatment $a$. Suppose we have summary outcomes $y_{jk}$ of treatment $k$ in study $j$. This summary outcome may be, for example, HRs, or RMST values. The standard NMA model is written as in Equation~\ref{eq:standardNMA1}-\ref{eq:standardNMA2}. In Equation~\ref{eq:standardNMA1}, $\pi_{Agg}$ is a suitable likelihood for the aggregate data, and $\theta_{jk}$ represents the expected summary outcome of treatment $k$ in study $j$. The link function $g$ serves to transform $\theta_{jk}$ onto the linear predictor scale. In Equation~\ref{eq:standardNMA2}, $\mu_j$ and $\delta_{jk}$ are study-specific intercepts and study-specific relative effect of treatment $k$ versus the reference treatment.

\begin{align}
    y_{jk} &\sim \pi_{Agg}(\theta_{jk}) \label{eq:standardNMA1} \\
    g(\theta_{jk}) &= \mu_j + \delta_{jk} \label{eq:standardNMA2}
\end{align}

There are two types of NMAs: fixed effect (FE) and relative effects (RE) NMAs. In an FE NMA, $\delta_{jk} = d_{1k} = d_k$, with $d_1 = 0$. In an RE NMA, $\delta_{jk} \sim N(d_k, \tau^2)$ for the heterogeneity variance $\tau^2$, with $\delta_{j1} = d_1 = 0$. \\

The standard NMA model assumes that any effect modifiers, i.e covariates that alter the relative effect on a given scale of an active treatment versus control, are balanced across populations. While this can often be a valid assumption, methods such as Matching-Adjusted-Indirect-Comparisons (MAICs), Simulated Treatment Comparisons (STCs), and ML-NMR have saught to relax this assumption by using IPD from at least one of the studies in a population.

\section{The General Multilevel Network Meta-Regression}
The derivation in this chapter outlines the work of~\cite{phillippo2024}. Under an NMA framework, there are $J$ RCTs investigating a subset $K_j \subset K$ $(j = 1,\ldots,J)$ treatments. In this project, $|K_j| = 2 \ \forall \ j$. Ideally, all IPD from the RCTs in a network would be available, however this is seldom the case. Certainly in a survival analysis context, often published KM curves accompanied by aggregate data are the only available data. From the perspective of a pharmaceutical company working on a Heatlth Technology Assessment (HTA) submission, the likely scenario is that they have data from their own RCT comparing their drug with a key comparator, and then published KM curves and covariate summaries. Consider first the standard IPD network meta-regression (NMR), as given in Defintion~\ref{def:mlnmrIPD}.

\begin{definition}[label=def:mlnmrIPD]{General IPD Meta-Regression Model}
    LLet $y_{ijk}$ be the IPD outcome for individual $i = 1, \ldots, N_{kj}$ in study $j$ recieving treatment $k \in K_j$ given the likelihood distribution $\pi_{Ind}(\theta_{ijk})$, where $N_{kj}$ is the number of patients on treatment $k$ in study $j$.
    \begin{align*}
        y_{ijk} &\sim \pi_{Ind}(\theta_{ijk}) \\
        g(\theta_{ijk}) &= \mu_j + x^T_{ijk}(\beta_1 + \beta_{2,k}) + \gamma_k \\
                        &= \eta_{jk}(x_{ijk})
    \end{align*}
    Here, $g$ links the likelihood parameter $\theta_{ijk}$ to $\eta_{jk}(x_{ijk})$. The $\mu_j$ are study-specific intercepts, and $\beta_1$, $\beta_{2,k}$ are regression coefficients for prognostic and effect-modifying covariates respectively. Additionally, the $\gamma_k$ are individual-level treatment effects. For the reference treatment, $\beta_{2,1} = \gamma_1 = 0$.
\end{definition}

It is clear to see how the model in Definition~\ref{def:mlnmrIPD} extends Equations~\ref{eq:standardNMA1}-\ref{eq:standardNMA2}.  Indeed, the model in Definition~\ref{def:mlnmrIPD} is based only on IPD. But the aim is to expand the model in order to incorperate aggregate data. By integrating the individual conditional likelihood function over the joint covariate distribution from an aggregate study, we obtain an individual marginal likelihood function. Compared to the conditional likelihood, the marginal likelihood describes the likelihood where outcomes are known, but covariates are not. In the context of survival analysis, the outcomes are the event and censoring times. \\

Let $\xi = \{\mu_j, \beta_1, \beta_{2,k}, \gamma_k | \forall \ j,k \}$ be the parameter space. Denote the individual conditional likelihood function by $L_{ijk|x}^{\text{Con}}(\xi; y_{ijk}, x_{ijk})$. The form of $L_{ijk|x}^{\text{Con}}$ depends on $\pi_{Ind}$, $g$, and $\eta_{jk}$.

Integrating $L_{ijk|x}^{\text{Con}}(\xi; y_{ijk}, x)$ over $f_{jk}$, we obtain the desired individual marginal likelihood function, as given in Equation~\ref{eq:marg}. 

\begin{equation}
    L_{ijk}^{\text{Mar}}(\xi; y_{ijk}) = \int_{\mathfrak{X}} L_{ijk|x}^{\text{Con}}(\xi; y_{ijk}, x)f_{jk}(x)dx
    \label{eq:marg}
\end{equation}

Here, $\mathfrak{X}$ is the support of $x$. It is clear from Equation~\ref{eq:marg} that $L_{ijk}^{\text{Mar}}$ does not depend on $x$ once the integral on the RHS of Equation~\ref{eq:marg} is evaluated. Let $i$ be an individual on treatment $k$ in study $j$ with outcome $y_{ijk}$. If we do not know the covariate vector for $i$, $x_{ijk}$, but we do know $f_{jk}$, then we know that the likelihood contribution of $i$ is given by Equation~\ref{eq:marg}. \\

It is likeliy that a closed-form of Equation~\ref{eq:marg} does not exist. Quasi Monte-Carlo methods can be used to evalauate the integral. We take a set of $N$ integration points, $\hat{x}$ from $f_{jk}$, and approximate the integral using Monte-Carlo methods (see Section~\ref{qmcSec}), giving Equation~\ref{eq:margApprox}.

\begin{equation}
    L_{ijk}^{\text{Mar}}(\xi; y_{ijk}) \approx \frac{1}{N}\sum_{\hat{x}}L_{ijk|x}^{\text{Con}}(\xi;y_{ijk},x)
    \label{eq:margApprox}
\end{equation}

Consider now a summary outcome $\hat{y_{jk}}$ aggregated over all individuals on treatment $k$ in study $j$. Each individual $i$ in on treatment $k$ in study $j$ contributes to the aggregate likelihood. Let $y_{ijk}$ denote the observed value of this summary measure for individual $i$. The aggregate maginal likelihood function is then the product of these $y_{ijk}$ up to a normalising constant, as in Equation~\ref{aggMargLik}.

\begin{equation}
    L_{\hat{jk}}^{\text{Mar}} \propto \prod_{i = 1}^{N_{jk}}L_{ijk}^{\text{Mar}}(\xi; y_{ijk})
    \label{aggMargLik}
\end{equation}

The full, general ML-NMR model is then given by combining the individual and aggregate level components, as in Definition~\ref{mlnmrdef}.

\begin{definition}[label=mlnmrdef]{General ML-NMR Model}
    IIndividual:
    \begin{align}
        L_{ijk|x}^{\text{Con}}(\xi;y_{ijk},x_{ijk}) &= \pi_{\text{Ind}}(y_{ijk}|\theta_{ijk}) \\
        g(\theta_{ijk}) &= \eta_{jk}(x_{ijk}) = \mu_j + x_{ijk}^T(\beta_1 + \beta_{2,k}) + \gamma_k \label{mlnmragg}
    \end{align}
    Aggregate:
    \begin{align}
        L_{ijk}^{\text{Mar}}(\xi; y_{ijk}) &= \int_{\mathfrak{X}} L_{ijk|x}^{\text{Con}}(\xi; y_{ijk}, x)f_{jk}(x)dx \label{mlnnmrint}\\
        L_{\hat{jk}}^{\text{Mar}} &\propto \prod_{i = 1}^{N_{jk}}L_{ijk}^{\text{Mar}}(\xi; y_{ijk})
    \end{align}
\end{definition}

Under a Bayesian framework, priors are placed on $\mu_j$, $\beta_1$, $\beta_{2,l}$, and $\gamma_k$. Uninformative priors were used.

\section{Survival ML-NMR}
The ML-NMR method lends itself well to survival outcomes. Each study in the network reports a pair $y_{ijk} = \{t_{ijk}, c_{ijk}\}$ consisting of outcome times $t_{ijk}$ and censoring indicators $c_{ijk}$, either from IPD from an RCT or reconstructed IPD. For IPD studies, the covariates $x_{ijk}$ will naturaly be avialble, but for aggregate studies (those for which pseudo-IPD has been re-created), only the joint covariate distribution of covariates at baseline, $f_{jk}$ is available. The idea for deriving the survival ML-NMR model is to include the $y_{ijk} = \{t_{ijk}, c_{ijk}\}$ in Definition~\ref{mlnmrdef}. \\

With some covariates, the pseudo-IPD can include a recreated covariate column for each individual. For example, if the proportion of male patients is given for a particular study, pseudo-patients can be randomly assigned to be male or female such that the number matches that reported in the trial. This is not possible with covariates such as age, where a distribution is usully reported. The proportion of male patients is somewhat unique in this case. It would likely not be statistically sound to do this with a variable such as the site of any metastises. Despite this being a count variable in the same way, the site of metastises may have more influence on survival times than the sex of a patient, and therefore randomly assigning patients to have a given metastises based on the aggregate data may not align with what the original data found. \\ 

The censoring indicator for patient $i$ in study $j$ on treatment $k$ is defined as in Equation~\ref{censInd}. In practice, the censoring indicator can be defined as $0$ meaning the individual is censored, or $1$ denotes censoring, as in Equation~\ref{censInd}. For this project, as all the data was digitised, the censoring indicator was already consistent across the data in the network, but when using an IPD study, one must take care to ensure the correct definition of censoring is consistent across all the studies. 

Let $S_{jk}(t|x)$ and $h_{jk}(t|x)$ be the survival and hazard functions at time $t$ conditional on the covariates $x$. Then the individual conditional likelihood contributions for each time $t_{ijk}$ in the IPD studies are given by

\begin{equation}
    L_{ijk|x}^{Con}(\zeta;t_{ijk},c_{ijk},x_{ijk}) = S_{jk}(t_{ijk}| x_{ijk})h_{jk}(t_{ijk}|x_{ijk})^{c_{ijk}}
\end{equation}

The forms of $S$ and $h$ depend on the specific survival models chosen. In this dissertation, only parametric models were considered for $S$ and $h$. Starting from Equation~\ref{eq:marg}, the marginal likelihood equations for each event/censoring time in the aggregate data studies can be derived. Substituting $y_{ijk} = \{t_{ijk}, c_{ijk}\}$ into $L_{ijk|x}^{\text{Con}}(\xi; y_{ijk}, x)$ results in Equation~\ref{eq:survint}.

\begin{align}
    L_{ijk}^{\text{Mar}}(\xi; y_{ijk}) &= \int_{\mathfrak{X}} L_{ijk|x}^{\text{Con}}(\xi; y_{ijk}, x)f_{jk}(x)dx \\ 
    L_{ijk}^{\text{Mar}}(\xi; t_{ijk}, c_{ijk}) &= \int_{\mathfrak{X}} L_{ijk|x}^{\text{Con}}(\xi; t_{ijk}, c_{ijk}, x)f_{jk}(x)dx \\
                                       &= \int_{\mathfrak{X}} S_{jk}(t_{ijk}| x_{ijk})h_{jk}(t_{ijk}|x_{ijk})^{c_{ijk}} f_{jk}(x)dx\label{eq:survint}
\end{align}

As with Equation~\ref{eq:margApprox}, Equation~\ref{eq:survint} can be evalueated with quasi-Monte Carlo integration to obtain Equation~\ref{lijk}. 

\begin{equation}
    L_{ijk}^{\text{Mar}}(\xi; t_{ijk}, c_{ijk}) = \frac{1}{N}\sum_{\hat{x}}S_{jk}(t_{ijk}|\hat{x})h_{jk}(t_{ijk}|\hat{x})^{c_{ijk}}
    \label{lijk}
\end{equation}

\section{Quasi Monte-Carlo}\label{qmcSec}
Quasi-Monte-Carlo (QMC) is a method for efficient numerical integration. Equation~\ref{mcProblem} gives the form of a Mote-Carlo integration problem for a real integrable function $f$ over the $s$-dimensional hypercube $I^s = [0,1]^s$. QMC differs from standard Monte-Carlo (MC) integration in the way the $x_i$ are chosen for which $f$ is evaluated at. 

\begin{equation}
    \int_{[0,1]^S}f(u)du \approx \frac{1}{N}\sum_{i = 1}^{N}f(x_i)
    \label{mcProblem}
\end{equation}

Regular MC integration uses random numbers\footnote{Strictly speaking, these are pseudo-random numbers when considering the implementation of this method.}, whereas QMC uses a low-discrepencey sequence, which is a sequence such that for all $N$, the set $P = \{x_1, \ldots, x_N\}$ has low discrepency, as definted in Definition~\ref{discrepency}.

\begin{definition}[label=discrepency]{Low-Discrepency Sequence}
    CConsider the set $P = \{x_1, \ldots, x_N\}$. The \textbf{discrepency} of $P$, $D_N(P)$ is defined as in Equation~\ref{def:descrep}.

    \begin{equation}
        D_N(P) = \sup_{B \in J}\left|\frac{A(B;P)}{N} - \lambda_s(B)\right|
        \label{def:descrep}
    \end{equation}

    Here, $A(B;P)$ is the number of points in $P$ that fall into $B$. $\lambda_s$ is the s-dimension Lebasgue measure, and $J$ is the set of $s$-dimensional intervals as in Equation~\ref{sbox}, where $0 \leq a_i < b_i \leq 1$.
    
    \begin{equation}
        \prod_{i = 1}^{s} [a_i,b_i) = \{x \in \mathbb{R}^s | a_i \leq x_i < b_i \}
        \label{sbox}
    \end{equation}
\end{definition}

In particular, \verb|multinma| uses a Sobol sequence~\cite{sobol} to sample $\hat{x_{jk}}$ from the covariate distribution $f_{jk}$~\cite{phillippo2020}. For each covariate in the analysis, the Sobol sequence generates points in $I^s = [0,1]^s$ with a dimension for each covariate. The Smirnov transform (Definition~\ref{smirnov}) is then used to transform the points generated by the Sobol sequence to the required distribution. Figure~\ref{rndPlot} presents the sample points in $[0,1]^2$ generated from sampling the normal distribution. Figure~\ref{sblPlot} presents the sample points from the Sobol sequence in the same region. The Sobol sequence clearly samples the space ina more efficient way. There are some fairly large gaps with no points in the random plot, but these are not present in the Sobol plot.  

\begin{definition}[label=smirnov]{Smirnov Transformation}
    LLet $X \in \mathbb{R}$ be any random variable. Then the random variable $F_X^{-1}(U)$ has the same distribution as $X$. Here, $F_X^{-1}$ is the generalised inverse of the cumulative distribution function $F_X$ of $X$, as defined in Equation~\ref{eq:invF} for all $p \in [0,1]$. Further, $U \sim Uniform[0,1]$.

    \begin{equation}
        F^{-1}(p) = \inf \{x \in \mathbb{R} | F(x) \geq p \}
        \label{eq:invF}
    \end{equation}
\end{definition}

\begin{figure}
    \centering
    \includegraphics[width = \textwidth]{../figures/rnd.png}
    \caption{Random sample of the unit square from a normal distribution}
    \label{rndPlot}
\end{figure}

\begin{figure}
    \centering
    \includegraphics[width = \textwidth]{../figures/sobol.png}
    \caption{Random sample of the unit square from a Sobol sequence}
    \label{sblPlot}
\end{figure}

\section{Population-average estimates}
Returning to ML-NMR, recall $d_{ab}$ is the relative effect of treatment b versus treatment a. Let $d_{ab(P)}$ be the population-average relative effect of b versus a in population P. $d_{ab(P)}$ can be calculated as in Equation~\ref{popavd}. 

\begin{align}
    d_{ab(P)} &= \int_{\mathfrak{X}}(\eta_{(P)b}(x) - \eta_{(P)a}(x))f_{(P)}(x)dx \label{popavgdInt} \\
              &= \gamma_b - \gamma_a + \bar{x}^T_{(P)}(\beta_{2,b}-\beta_{2,a}) \label{popavd}
\end{align}

For this dissertation, The treatment effect measures considered were the RMST and median OS. The survival and hazard functions were also analysed but were not primary outcomes for the NMA.

\subsection{Survival function}
Let $\bar{S}_{(P)k}(t)$ be the population-average marginal survival probability of treatment $k$ in population $P$ at time $t$. $\bar{S}_{(P)k}(t)$ is obtained by integrating $S_{(P)k}(t|x)$ over $f_{(P)}(x)$, as in Equation~\ref{popavgS}.

\begin{equation}
    \bar{S}_{(P)k}(t) = \int_{\mathfrak{X}}S_{(P)k}(t|x)f_{(P)}(x)dx
    \label{popavgS}
\end{equation}

\subsection{Hazard function}
The population-average marginal hazard function and cumulative hazard function are given by Equation~\ref{popavghaz}, and Equation~\ref{popavgcumhaz}, respectively.

\begin{align}
    \bar{h}_{(P)k}(t) &= \frac{\int_{\mathfrak{X}}S_{(P)k}(t|x)h_{(P)k}(t|x)f_{(P)k}(x)}{\bar{S}_{(P)k}(t)} \label{popavghaz}\\
    \bar{H}_{(P)k}(t) &= -\log(\bar{S}_{(P)k}(t)) \label{popavgcumhaz}
\end{align}

\subsection{RMST}
Let $x$ be some time horizon. The population-average marginal RMST follows from Definition~\ref{def:rmst}. By default, $\tau$ is the time such that $S(\tau) = 0$.

\begin{equation}
    RMST_{(P)k}(\tau) = \int^{\tau}_{0} \bar{S}_{(P)k}(t)dt.
\end{equation}

\subsection{Median OS}
In general, the $100\alpha\%$ quantile is obtained by solving Equation~\ref{survQuant} for $\alpha \in [0, 1]$.

\begin{equation}
    \bar{S}_{(P)k}(t) = 1 - \alpha.
    \label{survQuant}
\end{equation}

Since the median OS is a special case of this with $\alpha = \frac{1}{2}$, the population-average marginal median OS, $m$ is estimated by Equation~\ref{popavgmed}. 

\begin{equation}
    \bar{S}_{(P)k}(m) = \frac{1}{2}
    \label{popavgmed}
\end{equation}

\section{Treatment Ranking}
The other result considered by this NMA was the treatment ranking. This was done through the Surface Under the Cumulative Ranking Curve (SUCRA) values for each treatment to determine an overall ranking of each treatment. SUCRA values are in the range $0\%$ (worst treatment) to $100\%$ (best treatment)~\cite{mbuag}.

\section{Model Selection and Convergence}
Model selection was based primarily on the Leave-One-Out Information Criterion (LOOIC) score (Defintion~\ref{def:looic}). The LOOIC was calculated using the \verb|LOO| R package~\cite{loo} for each model. The model with the lowest LOOIC score was selected as the best fitting model. The Deviation Information Criterion (DIC) score (Definition~\ref{def:dic}) was used a secondary selection criterion. Again, models with lower DIC indicated better fit.

\begin{definition}[label=def:looic]{Leave-One-Out Information Criterion}
    BBegin by defining the expected log pointwise predictive density (elpd) as in Equation~\ref{elpd} for data $y$,
    
    \begin{equation}
        elpd_{loo} = \sum_{i = 1}^n \log(p(y_i | y_{-i}))
        \label{elpd}
    \end{equation}
    
    Here, $p(y_i|y_{-i})$ is the LOO predictive density given the data $y$ with the $i^{th}$ data point removed.
    
    \begin{equation}
        p(y_i | y_{-i}) = \int p(y_i|\theta)p(\theta|y_{-i})d\theta
        \label{loopd}
    \end{equation}
    
    Define the \textbf{Leave One Out Information Criterion (LOOIC)} score as in Equation~\ref{looic}.
    
    \begin{equation}
        LOOIC = -2elpd_{loo}
        \label{looic}
    \end{equation}
\end{definition}

\begin{definition}[label=def:dic]{Deviation Information Criterion (DIC)}
    LLet $y$ be the data on which a model was fitted. Further, let $\theta$ and $p(y|\theta)$ be the unknown paramaeters and likelihood function respectively. Let $E(\theta) = \bar{\theta}$ be the expectationi of $\theta$. The deviance is defined as in Equation~\ref{eq:dev}. 

    \begin{equation}
        D(\theta) = -2\log(p(y|\theta)) + C
        \label{eq:dev}
    \end{equation}

    Here, $C$ is an unknown constant. $C$ cancels in all model comparisons so is of no importance, but is included here for completeness. Further, the effective number of parameters is given by Equation~\ref{pd}.

    \begin{equation}
        p_D = \bar{D(\theta)} - D(\bar{\theta})
        \label{pd}
    \end{equation}

    The \textbf{Deviation Information Criterion (DIC)} is then given by Equation~\ref{dic}.

    \begin{equation}
        DIC = p_D + \bar{D(\theta)}
        \label{dic}
    \end{equation}
\end{definition}

The LOOIC score was preferred over the DIC score as the DIC score is based on a point estimate, and this NMA was performed in a Bayesian framewwork. As the LOOIC score is using the posterior distribution, it is a fully Bayesian statistic. Traditionally, the DIC score has been used due to the extra computational steps required for the LOOIC score. However,~\cite{vehtari} implented Pareto-Smoothed Importance Sampling (PSIS) in the R package \verb|loo|, which enables easy, and stable, computation of the LOOIC score from the NMA models. 
