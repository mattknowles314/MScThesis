\chapter{Conclusion and Discussions}\label{conclusions}

\section{Conclusion}

\section{Discussion and Further Work}
The results obtained in this dissertation align with an NMA conducted by~\cite{gresham2014}, which also found GEM-NAB and GEM-CAP offered improvements in OS compared to GEM monotherapy. This NMA included fewer studies than that of~\cite{gresham2014}, so future work to expand the number of trials would be good to solidfy the place of GEM-NAB and GEM-CAP as the best treatment options for advanced/pancreatic cancer. \\

In 2017, NICE published TA476~\cite{TA476}, which recommended GEM-NAB as an option for untreated metastatic pancratic cancer in adults only if other combination chemotherapies are unsuitable and the patient would otherwise recieve GEM monotherapy. The TA claimed that GEM-NAB was more effective in increasing than GEM monotherapy, similarly effctive to GEM-CAP, but was less effective than FOLFIRINOX. The comparison to GEM-CAP was stated as ``uncertain''.


Given the poor prognosis of pancreatic cancer, even a small improvement in median OS of a couple of months is of immense emotional value to patients and their families, and should not be overlooked for looking like small improvements out of context.

\section{Considerations for the ISPOR Good Practice Task Force}
ISPOR developed a 26-item questionnaire for assessing the credibility of an NMA~\cite{jansengp}. This NMA was performed inline with these practices. While each question is not answered individually here, the themes of the guidance, and how this NMA aligns with it, are discussed. \\

The first set of questions in the guidance concerns the evidence base. This NMA was performed on a fully-connected network of evidence (Figure~\ref{fig:osnet}), and included no poor-quality studies. Indeed, the study populations and trial characteristics were similar, meaning there was no systematic differences in treatment effect modifiers across the comparisons. The only aspect of this NMA that could be considered not to follow these guidelines was that not all available RCTs were included. The Greshem study, for example included 23 studies obtained by seaching several databases. This NMA was not conducted based on results of a systematic literature review or database search. Studies were selected for this NMA based on a brief literature search for trials comparing GEM with another therapy. Since all the KM curves from published papers had to be digitised, which takes a considerable amount of time, there was always to be a limit on how many studies could be included. \\

The second set of questions concerns the analysis. No na\"ive comparisons were made, which preserve within-study randomisation. As there were no cases of both direct and indirect evidence for any treatments, questions eight and nine were not deemed relevant. Question ten concerns imbalance of the distribution of effect modifiers, and how this was accounted for. Since the ML-NMR is a meta-regression model, this was directly accounted for. In terms of FE and RE models, both were fit, and the best fitting model selected in terms of robust selection statistics. Since the studies included in this NMA were not diverse in terms of methodology, FE models were deemed to be clinically appropriate. The guidance generally reccomends RE models, but it was deemed clinically appropriate to consider FE models in this case. Were more trials to be included, more consideration would need to be given to the similarity assessment to determine the suitability of FE models. \\

The third set of questions relates to the reporting quality. while all the studies used, and indeed the associated KM curves were presented, the actual TTE data was not presented. This is due to the form of the data, although it is available within \verb|PCNMA| R package. Individual study results were provided in Figures~\ref{fig:pred_survbc}-~\ref{fig:pred_medianbc}. Considerations did not have to be made for direct and indirect comparisons since there were no closed loops. Rankings were reported to address the main project aim, and pairwise comparisons were reported. In particular, the pairwise comparison between GEM-CAP and GEM-NAB was important to clarify the uncertainty mentioned by NICE in NG85. No consideration was given to the effect of important patient characteristics due to the homogeneity in the trial populations and further because of a lack of IPD available for this study. It is not sound for those involved in the study to assess the fairness of the conclusions and interpretation, but every attempt was made to perform this NMA with integrity and interpret the results in line with the evidence. \\

