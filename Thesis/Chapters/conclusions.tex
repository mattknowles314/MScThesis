\chapter{Conclusion and Discussions}\label{conclusions}

\section{Conclusion}
FOL provides improved OS compared to GEM-NAB and GEM-CAP for the treatment of advanced/metastatic pancreatic cancer. FOL outperformed all treatments in terms of RMST and median OS in all study populations considered in the network of evidence. Based on the survival time ratio, FOL and GEM-NAB provided statistically signficant OS benefit compared to GEM. GEM-CAP also provided better OS benefit than GEM, but not significantly. Conversely, GEM-SOR and GEM-IRI provided worse OS benefit compared to GEM, but not signficantly worse. \\

From a clinical perspective, this means FOL should always be considered first for patients who are able to tolerate it, followed by GEM-NAB, and GEM-CAP. GEM should be used for patients who cannot tolerate FOL or GEM-NAB, as GEM-AXI, GEM-IRI, or GEM-PEM do not provide better OS, and may be too toxic to make the treatment worthwhile compared to GEM. While this NMA did not consider toxicitiy, GEM is known to be less toxic than combination therapies. \\

The second conclusion of this NMA was that GEM-NAB does provide better OS for advanced/metastatic pancreatic cancer than GEM-CAP, but the difference is not significant. This was a consistent conclusion reached in the base case and both sensitivity analyses. The choice of GEM-NAB or GEM-CAP for a patient should be at the discretion of the patient's oncologist.

\section{Discussion}
In 2017, NICE published TA476~\cite{TA476}, which recommended GEM-NAB as an option for untreated metastatic pancratic cancer in adults only if other combination chemotherapies are unsuitable and the patient would otherwise recieve GEM monotherapy. The TA claimed that GEM-NAB was more effective in increasing OS than GEM monotherapy, similarly effctive to GEM-CAP, but was less effective than FOL. The comparison to GEM-CAP was stated as ``uncertain''. This NMA found this to indeed be the case, but with more clarity on the comparison of GEM-NAB and GEM-CAP. GEM-NAB is more effective than GEM-CAP, but not significantly. \\ 

While the results of this NMA were conclusively in favour of FOL, the fact that patients in the only trial comparing GEM and FOL were capped at age 76 should not be forgotten. The median age was similar to other studies, and so it is not likely that the age had a signifcant effect on the performance of FOL, but it would be useful to include an RCT in the network that did not have any exclusion criteria. No such trial appears to exist at present. Two studies:~\cite{matsumoto}, and~\cite{chung} were found, but these studies considered patients who had not responded to GEM (Chung, Matsumoto) and GEM or GEM-NAB (Chung). It would therefore not have been appropriate to include these studies in the NMA. This eligibilty criteria on age in the Conroy study was not deemed to detract from the clinical validity of these results. One thing that should be considered, which was also pointed out in the discussion of the Nichetti NMA, is that GEM-NAB is administered to a wider patient population. This NMA has proved that the ML-NMR method could plug this gap in the literature if an IPD study was available. If IPD were available, these covariates could easily be included. \\

It is reccomended for any future work to either increase the size of the network of evidence, or include more studies for current comparisons. This NMA is based on fewer assumptions than a standard HR-based survival NMA, but corroborated the findings of several meta analyses, as discussed in Chapter~\ref{litChap}. Indeed, FOL, followed by GEM-NAB are clearly the best treatments for advanced/metastatic pancreatic cancer. Based on the results of this NMA and the literature explored in Chapter~\ref{litChap}, it is likely that the current treatment landscape for pancreatic cancer is well understood, and there is little to no debate to be had over the best treatments. It is perhaps more wise to invest time and energy on exploring other options for treating this horrible disease, such as surgery. Alternatively, given the number of excess deaths coming from the inability to diagnose pancreatic cancer early, some form of screening process may result in catching cases earlier, giving clinicans more options on how to treat patients. This was alluded to in 1999 by~\cite{dimagno}, as discussed in Chapter~\ref{backChap}, and should be given full consideration by NICE and the NHS. This would of course require full economic evaluation. \\

In terms of the economics of GEM-NAB and FOL, the Incramental Cost Effectiveness Ratio (ICER) for GEM-NAB compared to GEM is between $£41,000$ and $£46,000$ per Qality-Adjusted Life Year (QALY) gained~\cite{TA476}. The same NICE TA noted that FOL dominated GEM-NAB\footnote{i.e, FOL was both more effective, and cheaper}, making it the preferred treatment. In turn, GEM-CAP dominated GEM-NAB. This NMA casts some doubt on this as GEM-NAB demonstrated improved efficacy to GEM-CAP. A Cost-Effectiveness Model (CEM) of neoadjuvant FOL compared to GEM-NAB from 2021 found the ICER of FOL compared to GEM-NAB to be $\$60,856.47$ ($\sim £47,392$ based on the July 2024 USD/GBP exchange rate)~\cite{ingram}. A different CEM that took a US perspective and found FOL that the increase in effectiveness associated with FOL was also associated with a $\$40,831$ increase in the cost when compated to GEM-NAB~\cite{kharat}. The authors found the ICER of FOL compared to GEM-NAB to be $\$226,841$ ($\sim £176,614$ based on the July 2024 USD/GBP exchange rate), making GEM-NAB a more cost-effective treatment. It should be noted that the Kharat CEM was based on resected pancreatic cancer patients, and the Ingram study considered borderline resectable/locally advanced pancreatic cancer patients, making it more appropriate for this analysis. \\

Given the poor prognosis of pancreatic cancer, even a small improvement in median OS of a couple of months is of immense emotional value to patients and their families, and should not be overlooked for looking like small improvements out of context.

\section{Considerations for the ISPOR Good Practice Task Force}
ISPOR developed a 26-item questionnaire for assessing the credibility of an NMA~\cite{jansengp}. This NMA was performed inline with these practices. While each question is not answered individually here, the themes of the guidance, and how this NMA aligns with it, are discussed. An answer to each question individually is available in Appendix~\ref{isporqs}\\

The first set of questions in the guidance concerns the evidence base. This NMA was performed on a fully-connected network of evidence (Figure~\ref{fig:osnet}), and included no poor-quality studies. Indeed, the study populations and trial characteristics were similar, meaning there was no systematic differences in treatment effect modifiers across the comparisons. The only aspect of this NMA that could be considered not to follow these guidelines was that not all available RCTs were included. The Greshem study, for example included 23 studies obtained by seaching several databases. This NMA was not conducted based on results of a systematic literature review or database search. Studies were selected for this NMA based on a brief literature search for trials comparing GEM with another therapy. Since all the KM curves from published papers had to be digitised, which takes a considerable amount of time, there was always to be a limit on how many studies could be included. Since pancreatic cancer treatments have not changed much, the fact the studies included in this NMA were reported between 2004 and 2015 was not deemed to rended the results of the NMA inconclusive in today's treatment landscape. \\

The second set of questions concerns the analysis. No na\"ive comparisons were made, which preserve within-study randomisation. As there were no cases of both direct and indirect evidence for any treatments, questions eight and nine were not deemed relevant. Question ten concerns imbalance of the distribution of effect modifiers, and how this was accounted for. Since the ML-NMR is a meta-regression model, this was directly accounted for. In terms of FE and RE models, both were fit, and the best fitting model selected in terms of robust selection statistics. Since the studies included in this NMA were not diverse in terms of methodology, FE models were deemed to be clinically appropriate. The guidance generally reccomends RE models, but it was deemed clinically appropriate to consider FE models in this case. Were more trials to be included, more consideration would need to be given to the similarity assessment to determine the suitability of FE models. \\

The third set of questions relates to the reporting quality. while all the studies used, and indeed the associated KM curves were presented, the actual TTE data was not presented. This is due to the form of the data, although it is available within \verb|PCNMA| R package. Individual study results were provided in Figures~\ref{fig:pred_survbc}-~\ref{fig:pred_medianbc}. Considerations did not have to be made for direct and indirect comparisons since there were no closed loops. Rankings were reported to address the main project aim, and pairwise comparisons were reported. In particular, the pairwise comparison between GEM-CAP and GEM-NAB was important to clarify the uncertainty mentioned by NICE in NG85. No consideration was given to the effect of important patient characteristics due to the homogeneity in the trial populations and further because of a lack of IPD available for this study. It is not sound for those involved in the study to assess the fairness of the conclusions and interpretation, but every attempt was made to perform this NMA with integrity and interpret the results in line with the evidence. \\

