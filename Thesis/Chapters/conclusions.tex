\chapter{Conclusion and Discussions}\label{conclusions}

\section{Conclusion}
FOL provides improved OS compared to GEM-NAB and GEM-CAP for the treatment of locally advanced/metastatic pancreatic cancer. FOL outperformed all other treatments in terms of both RMST and median OS in all study populations considered. Based on the survival time ratio, FOL and GEM-NAB provided statistically signficant OS benefit compared to GEM. GEM-CAP also provided better OS benefit than GEM, but not significantly. Conversely, GEM-SOR and GEM-IRI provided worse OS benefit compared to GEM, but not signficantly worse. \\

From a clinical perspective, this means FOL should always be considered first for patients who are able to tolerate it, followed by GEM-NAB, and GEM-CAP. GEM should be used for patients who cannot tolerate FOL or GEM-NAB, as GEM-AXI, GEM-IRI, and GEM-PEM do not provide better OS, and may be too toxic to make the treatment worthwhile compared to GEM. While this NMA did not consider toxicitiy as an outcome, GEM is known to be less toxic than combination therapies. \\

The second conclusion of this NMA was that GEM-NAB does provide better OS for advanced/metastatic pancreatic cancer than GEM-CAP, but the difference is not significant. This was a consistent conclusion reached in the base case and both sensitivity analyses. The choice of GEM-NAB or GEM-CAP for a patient should be at the discretion of the patient's oncologist.

\section{Discussion}
In 2017, NICE published TA476~\cite{TA476}, which recommended GEM-NAB as an option for untreated metastatic pancratic cancer in adults only if other combination chemotherapies are unsuitable and the patient would otherwise recieve GEM monotherapy. The TA claimed that GEM-NAB was more effective in increasing OS than GEM monotherapy, similarly effctive to GEM-CAP, but was less effective than FOL. The comparison to GEM-CAP was stated as ``uncertain''. This NMA found this to indeed be the case, but with more clarity on the comparison of GEM-NAB and GEM-CAP. GEM-NAB is more effective than GEM-CAP, but not significantly. \\ 

While the results of this NMA were conclusively in favour of FOL, the fact that patients in the only trial comparing GEM and FOL were capped at age 76 should not be forgotten. The median age was similar to other studies, and so it is not likely that the age had a signifcant effect on the performance of FOL, but it would be useful to include an RCT in the network that did not have any exclusion criteria. No such trial appears to exist at present. Two studies:~\cite{matsumoto}, and~\cite{chung} were found, but these studies considered patients who had not responded to GEM (Chung, Matsumoto) and GEM or GEM-NAB (Chung). It would therefore not have been appropriate to include these studies in the NMA. This eligibilty criteria on age in the Conroy study was not deemed to detract from the clinical validity of these results. One thing that should be considered, which was also pointed out in the discussion of the Nichetti NMA, is that GEM-NAB is administered to a wider patient population. This NMA has proved that the ML-NMR method could plug this gap in the literature if an IPD study was available. If IPD were available, these covariates could easily be included. \\

It is reccomended for any future work to either increase the size of the network of evidence, or include more studies for current comparisons. This NMA is based on fewer assumptions than a standard HR-based survival NMA, but corroborated the findings of several meta analyses, as discussed in Chapter~\ref{litChap}. Indeed, FOL, followed by GEM-NAB are clearly the best treatments for advanced/metastatic pancreatic cancer. Based on the results of this NMA and the literature explored in Chapter~\ref{litChap}, it is likely that the current treatment landscape for pancreatic cancer is well understood, and there is little to no debate to be had over the best treatments. It is perhaps more wise to invest time and energy on exploring other options for treating this horrible disease, such as surgery. Alternatively, given the number of excess deaths coming from the inability to diagnose pancreatic cancer early, some form of screening process may result in catching cases earlier, giving clinicans more options on how to treat patients. This was alluded to in 1999 by~\cite{dimagno}, as discussed in Chapter~\ref{backChap}, and should be given full consideration by NICE and the NHS. This would of course require full economic evaluation. \\

In terms of the economics of GEM-NAB and FOL, the Incramental Cost Effectiveness Ratio (ICER) for GEM-NAB compared to GEM is between $£41,000$ and $£46,000$ per Qality-Adjusted Life Year (QALY) gained~\cite{TA476}. The same NICE TA noted that FOL dominated GEM-NAB\footnote{i.e, FOL was both more effective, and cheaper}, making it the preferred treatment. In turn, GEM-CAP dominated GEM-NAB. This NMA casts some doubt on this as GEM-NAB demonstrated improved efficacy to GEM-CAP. A Cost-Effectiveness Model (CEM) of neoadjuvant FOL compared to GEM-NAB from 2021 found the ICER of FOL compared to GEM-NAB to be $\$60,856.47$ ($\sim £47,392$ based on the July 2024 USD/GBP exchange rate)~\cite{ingram}. A different CEM that took a US perspective and found FOL that the increase in effectiveness associated with FOL was also associated with a $\$40,831$ increase in the cost when compated to GEM-NAB~\cite{kharat}. The authors found the ICER of FOL compared to GEM-NAB to be $\$226,841$ ($\sim £176,614$ based on the July 2024 USD/GBP exchange rate), making GEM-NAB a more cost-effective treatment. It should be noted that the Kharat CEM was based on resected pancreatic cancer patients, and the Ingram study considered borderline resectable/locally advanced pancreatic cancer patients, making it more appropriate for this analysis. \\

The ML-NMR method is clearly a powerful method for conducting population-adjusted NMAs. With R packages like \verb|multinma|, it is easy to implement and conduct statistically and clinically valid analyses. In the second SA, the convergence of the RE model was much better than the analyses that included more immature data, suggesting that the maturity of the survival data may have an impact on the convergence of the models. However, given that the model was only fit with one covariate, it would be wise to assess this in the presence of more IPD. A simulation study may be appropriate for assessing this situation. One thing that was not assessed was the sensitivity of the method to the choice of distribution for the likelihood, but this would also be interesting to assess in the future. The ML-NMR method works well with spline-based likelihood models, but these were not considered here due to the maturity of the data meaning good model fit was able to obtained with parametric survival models.\\

Pancreatic cancer remains a nasty disease, despite the advancements in treatments. More focus should be given to national screening programs in the future. Operable pancreatic cancer has much better long term survival, but it is only a ``fortunate'' few patients who find their cancer early enough to be operable. Given the poor prognosis of pancreatic cancer, even a small improvement in the long-term OS of a couple of months is of immense emotional value to patients and their families when removal of the tumour is not possible. The small improvement in OS given by the treatments evaluated in this NMA should not be overlooked for looking like small improvements compared with treatments for other cancers.

