\chapter{Conclusions and Recommendations}

The conclusions of the five studies included in the NMA are given below (with drug names abbreviated).

\begin{itemize}
    \item Colucci: ``GEM-CIS failed to demonstrate any improvements as a first-line treatment of advanced pancreatic cancer''
    \item Cunningham: ``GEM-CAP should be considered as one of the standard first-line options in locally advanced and metastatic pancreatic cancer''
    \item Kindler: ``GEM-AXI does not improve overall survival in advanced pancreatic cancer''
    \item Oettle: ``GEM-PEM therapy did not improve OS''
    \item Rocha Lima: ``IRI-GEM safely improved the tumour response rate compared with GEM but did not alter overall survival''
\end{itemize}

\section{Parametric NMA}
Based on guidance from NICE TSD 2~\cite{tsd2}, the fixed effects model was selected as the best fitting model due to having a lower DIC score. However it should be noted that the differences between the models were very small, with only 0.2 seperating the DIC scores. \textbf{The model showed no significant change in overall survival for any of the 5 comparators to GEM in the treatment of pancreatic cancer}. GEM-AXI, GEM-CAP, and GEM-PEM had positive treatment effects, although GEM-AXI and GEM-PEM were close to 0. GEM-CIS and GEM-IRI had slightly negative treatment effects, but were both close to 0. 

