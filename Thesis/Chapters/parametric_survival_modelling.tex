\chapter{Parametric Models for Survival Analysis}

This chapter discusses the parametric models commonly used in Survival Analysis. In particular, the seven parametric models recommended by NICE in TSD~\cite{tsd14}. All parametric model fitting for this project was performed in \verb|R| using the \verb|flexsurv| package~\cite{flexsurv}. The first section outlines how the \verb|flexsurv| package works.

\section{Model Setup}
The general model of a \verb|flexsurv| survival model takes the form 

\begin{equation}
    \label{fsurvmodel}
    f(t|\mu(\bf{z}), \bf{\alpha}(\bf{z})).
\end{equation}

Equation~\ref{fsurvmodel} gives the probability density for death at time $t \geq 0$. The \textit{mean} or \textit{location} of the distribution is given by $\mu = \alpha_0$. The remaining parameters, $\bf{\alpha^1} = (\alpha_1, \ldots, \alpha_R)$ are called \textit{ancillary} parameters. \\

Chapter~\ref{survchap} discussed AFT and PH models. Under the \verb|flexsurv| framework, if the hazard function, $h(t) = \frac{f(t)}{S(t)}$, can be factorised as 

\[
    h(t|\bf{\alpha}, \mu(\bf{z})) = \mu(\bf{z})h_0(t|\bf{\alpha}). 
\] 

Then we have a PH model. On the other hand, an AFT model would be written as

\[
    S(t|\mu(\bf{z}), \bf{\alpha}) = S_0(\mu(\bf{z})t/\bf{\alpha}).  
\]

All parameters may depend on $\bf{z}$, a vector of covariates. This is done through the link-transformed linear models

% MATT: THERE IS BOLD TEXT HERE THAT SHOULDNT BE!
\begin{align}
    \label{flexsurvMLE}
    g_0(\mu(\bf{z})) &= \gamma_0 + \bf{\beta}_0^Tz \nonumber \\
    g_r(\alpha_r(\bf{z})) &= \gamma_r + \bf{\beta}_{\gamma}^T\bf{z}
\end{align}

$g$ is usally chosen to be $log()$ if the parameter is positive, or the identity function if the parameter is unrestricted.\\

\verb|Flexsurv| contains several built-in models. Appendix~\ref{flexsurvDists} presents these models.

\section{Fitting Models}
Let $t_i$, $i \in \{1, \ldots, n\}$, be a sample of times from $n$ individuals. Define $c_i$ such that 

\[
    c_i = \begin{cases*}
        1 & \text{if $t_i$ is an observed death time} \\
        0 & \text{if $t_i$ is censored} 
    \end{cases*}  
.\]

Introduce $s_i$, which are delayed-entry times. This means for an individual $i$ who is delayed-entry, the survival time is only observed conditionally on individual $i$ having survived up to time $s_i$. $s_i = 0$ when there is no delayed-entry. \\

\subsection{Right Censoring}
In the case of right-censoring and nothing else, the likelihood for the parameters $\bf{\theta} = \{\bf{\gamma}, \bf{\beta}\}$ required in Equation~\ref{flexsurvMLE} is given by 

\begin{equation}
    \label{flexsurvRightCens}
    l(\bf{\theta}|\bf{t},\bf{c},\bf{s}) = \frac{\prod_{i:c_i=1} f_i(t_i) \prod_{i:c_i=0} S_i(t_i)}{\prod_{i} S_i(s_i)}
\end{equation}

\subsection{Interval Censoring}
In the case of interval-censoring, where the survival time is censored on $(t_i^{\text{min}}, t_i^{\text{max}})$, the likelihood for $\bf{\theta} = \{\bf{\gamma}, \bf{\beta}\}$ is 

\begin{equation}
    \label{flexSurvIntCens}
    l(\bf{\theta}|\bf{t^\text{min}}, \bf{t^\text{max}}, \bf{c},\bf{s}) = \frac{\prod_{i:c_i=1} f_i(t_i) \prod_{i:c_i=0} \left(S_i(t_i^{\text{min}}) - S_i(t_i^{\text{max}}) \right)}{\prod_{i} S_i(s_i)}
\end{equation}

Maximum Likelihood Estimation is performed in \verb|R| using the analytic derivatives of Equation~\ref{flexsurvRightCens} and/or Equation~\ref{flexSurvIntCens}. 


