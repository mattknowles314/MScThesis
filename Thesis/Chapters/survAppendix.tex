\chapter{Survival Distributions Reference}

This appendix outlines the mathematical definitions of the parametric and non-parametric survival models used in this project. In addition, some information on model selection is given. 

\section{Parametric Survival Models}\label{flexsurvDists}

\subsection{Exponential Model}
The survival function for the exponential model is given by \[S(t|\lambda) = \exp(-\lambda t)\] for $\lambda > 0$.

\subsection{Gamma}
The survival function for the gamma model is give by \[S(t|a,\mu) = 1 - \int_{0}^{t}\frac{x^{a-1}\exp(-x/ \mu)}{\mu^a \Gamma(a)}dx\] for $\mu > 0, a > 0$.

\subsection{Generalised Gamma}
The survival function for the generalised gamma model, by default, uses the Prentice parameterisation [REFERENCE]. The survival function is given by 

\[
S(t|\mu, \sigma, Q) = \begin{cases}
    S_G(\frac{\exp(Qw)}{Q^2}|\frac{1}{Q^2}, 1) & \text{if } Q > 0 \\
    S_L(t | \mu, \sigma) & \text{if } Q = 0 \\
    1 - S_G(\frac{\exp(Qw)}{Q^2}|\frac{1}{Q^2}, 1) & \text{if } Q < 0 \\
\end{cases}    
\]

Where $S_G, S_L$ are the survival functions gor the gamma and log-normal distributions respectively. In addition, $w = \frac{\log(t)-\mu}{\sigma}$. Where $\sigma > 0$.

\subsection{Gompertz}
The survival function for the Gompertz model is given by \[S(t|a,b) = \exp(-(b/a)\exp(at)-1)\] for $b > 0$.

\begin{remark}{Gompertz special case}
    TStrictly speaking, $a < 0$ is permitted, however, this causes the survival function to take the form $S(t|a,b) = \exp(b/a\exp(at)-1)$. We therefore have, using the fact that $\lim_{t \to \infty} \exp(at) = 0$,
    \begin{align*}
        S(t) &= \exp(b/a\exp(at)-1) \\
        \lim_{t \to \infty} S(t) &= \lim_{t \to \infty} \exp(b/a\exp(at)-1) \\
        \lim_{t \to \infty} S(t) &= \exp(-1).
    \end{align*}
    Therefore, $S(t)$ tending to a non-zero probability as $t \to \infty$ is clinically inplausible.
\end{remark}

\subsection{Log-Logistic}
The survival function for the log-logistic model is given by \[S(t|a,b) = \frac{1}{1+(t/b)^a}\] for $a > 0, b > 0$.

\subsection{Log-Normal}
The survival function for the log-normal model is given by \[S(t|\mu, \sigma) = 1 - \int_{0}^{t} \frac{1}{x\sigma\sqrt{2\pi}}\exp\left(-\frac{(\log x - \mu)^2}{2\sigma^2} \right)dx \] for $\sigma > 0$.

\subsection{Weibull}
The AFT Weibull parameterisation is used in this project. The survival function for the AFT Weibull is given by \[S(t|\mu,a)=\exp(-(t/\mu)^a)\]

\section{Selection of Survival Models}\label{model_selction}
Selecting the survival model for a given study is done for the GEM arm only, as this is the reference treatment. Akiakie's Information Criterion (AIC) [REFERENCE] is used.

\begin{definition}{Akaikie's Information Criterion}{aic}
    For a given set of models, the \textbf{Akaike's Information Criterion (AIC)} score is given by 
    \[
     AIC = 2k - 2\log(\hat{L}).   
    \]
    Where $k$ is the number of parameters in the model, and $\hat{L}$ be the maximised value of the likelihood function for the model.
\end{definition}

When selecting models based on AIC score, the lowest scoring model is considered the best fitting model. The intrinsic value of the scores is irrelevant, only their values relative to one another. Models with an AIC score of $\leq 2$ of the best fitting model are also considered well-fitting models.