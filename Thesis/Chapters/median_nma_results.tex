\chapter{NMA of Pancreatic Cancer Trials using Median OS}

This appendix presents NMA results using median OS as the outcome measure instead of 


\section{Parametric Models}

Table~\ref{paramNMAInputs} presents the input data used for the parametric NMA. 

\begin{table}[h]
    \center
    \begin{tabular}{llllllll}
    \hline
    Distribution      & Treatment & Median  & L95    & U95     & Study      & n   & SE      \\ \hline
    Gompertz          & GEM       & 8.5075  & 7.0869 & 9.9992  & Colucci    & 400 & 14.859 \\
    Generalised Gamma & GEM-CIS   & 7.3009  & 6.2111 & 8.5350  & Colucci    & 400 & 11.857 \\
    Log-Logistic      & GEM       & 5.9954  & 5.4301 & 6.6452  & Cunningham & 533 & 7.1564  \\
    Generalised Gamma & GEM-CAP   & 7.2947  & 6.5520 & 8.1846  & Cunningham & 533 & 9.6151  \\
    Exponential       & GEM       & 10.154  & 8.5389 & 12.230  & Kindler    & 632 & 23.675 \\
    Exponential       & GEM-AXI   & 9.8440  & 8.0901 & 11.928  & Kindler    & 632 & 24.613 \\
    Generalised Gamma & GEM       & 6.1225  & 5.4602 & 6.9002  & Oettle     & 565 & 8.7319  \\
    Gamma             & GEM-PEM   & 6.5691  & 5.9814 & 7.2575  & Oettle     & 565 & 7.7381  \\
    Gamma             & GEM       & 6.7650  & 5.9708 & 7.6333  & Rocha Lima & 360 & 8.0471  \\
    Weibull           & GEM-IRI   & 6.2512  & 5.3539 & 7.3042  & Rocha Lima & 360 & 9.4399  \\ \hline
    \end{tabular}
    \caption{Input data for the parametric NMA}
    \label{paramNMAInputs}
\end{table}

Figure~\ref{fig:forests_param} presntes the forest plots of the NMA conducted on median survival estimates. Both fixed and random effects models were fit. There was little difference between the two models, as shown by the selection statistics in Table~\ref{paramDIC}. Note that the forest plots in Figure~\ref{fig:forests_param} have been zoomed in to show the differences more clearly, the full figures are presented in Section~\ref{nmaappend}, along with further model diagnostic plots. The models were fit with uninformative priors of $N(0, 100^2)$, $N(0, 10^2)$, and $hN(0, 5^2)$ for the intercept, treatment and heterogeneity respectively. Here $hN$ is the half-normal distribution. The trace plots, as presented in Section~\ref{nmaappend} suggested good convergence for both models. 

\begin{figure}[h]
    \centering
    \begin{subfigure}[b]{0.45\textwidth}
        \centering
        \includegraphics[width=\textwidth]{../figures/FE-Zoomed.png}
        \caption{Fixed Effects forest plot for parametric models}
        \label{fig:FEForestParam}
    \end{subfigure}
    \hfill
    \begin{subfigure}[b]{0.45\textwidth}
        \centering
        \includegraphics[width=\textwidth]{../figures/RE-Zoomed.png}
        \caption{Random Effects forest plot for parametric models}
        \label{fig:REForestParam}
    \end{subfigure}
    \caption{Forest plots for the fixed and random effects models of survival}
    \label{fig:forests_param}
\end{figure}

\begin{table}[h]
    \center
    \begin{tabular}{lll}
    \hline
    Model          & pD  & DIC  \\ \hline
    Fixed Effects  & 6.5 & 13.1   \\
    Random Effects & 6.6 & 13.3 \\ \hline
    \end{tabular}
    \caption{Model selection statistics for the parametric NMA (Medain OS) models}
    \label{paramMedianDIC}
\end{table}