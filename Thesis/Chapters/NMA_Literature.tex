\chapter{A Review of Meta-Analyses for Pancreatic Cancer}\label{litChap}
Prior to conducting this NMA, the literature was searched to identify MAs (ideally NMAs) comparing treatments for pancreatic cancer that included either all, or a subset of, the treatments included in this NMA. The literature on meta-analyses, and in particular NMAs is quite spares. Unfortunately, as a result of this, an NMA using an ML-NMR approach could not be found. Six meta-analyses were found, using a range of methods.

\section{An Overview of MAs}
In~\cite{chen}, the authors used a frequentist model to analyse both survival and toxicity data between modified FLOFIRINOX, regular FOLFIRINOX, and GEM-NAB. Data from 22 studies were includeed. The results of this NMA suggest that GEM-NAB and FOLFIRINOX are similarly efficacious, but FOLORINOX provides superior OS compared to GEM than GEM-NAB does. The results suggested that the improved efficacy of FOLFIRINOX and GEM-NAB were statistically significant compared to GEM, but GEM-NAB and FOLFIRINOX were not significantly different. The Chen NMA also showed the toxicity profile of FOLFIRINOX and GEM-NAB is similar. In particular, nausea diarrhea were more frequently observed in patients treated with FOLFIRINOX than GEM-NAB. In addition, patients treated with GEM-NAB had slighlty higher risk of fatigue and anemia. \\

The NMA conducted in~\cite{lin} included 31 studies in an efficacy meta-analysis and 32 studies in a safety meta-analysis. Naturally, only the efficacy analysis was relevant for this NMA. As with the Chen NMA, the authors used a frequentist NMA, and assessed the HRs. The focus of the Lin NMA was the first-line treatment of advanced pancreatic cancer. The only treatment included in the Lin NMA that was also relevant to this NMA was FOLFIRINOX. They found that FOLFIRINOX-based therapy was the best treatment. \\

The Nichetti meta-analysis~\cite{nichetti} was not an NMA, but instead reconstructed KM data from seven studies, and then pooled the reconstructed IPD to compare the efficacy of FOLFIRINOX, NALIRIFOX, and GEM-NAB. Only phase III studies were included in this NMA. They found that FOLFIRINOX provided superior OS to GEM-NAB. They did validate their results with an NMA, although this was not reported in the main paper. The median OS of all treatments in this study was below 12 months. In terms of toxicity, the Nichetti NMA found NALIRIFOX had a favourable safety profile compared to FOLFIRINOX and GEM-NAB. \\

An NMA conducted by~\cite{gresham2014}, which included FOLFIRINOX, GEM-NAB, and GEM-CAP, found all three treatments to be associated with statisitically significant improvements in OS relative to GEM and several other treatments. The Gresham NMA was a Bayesian NMA for calculating survival outcomes. The primary result outcomes of their NMA were the HR and survival gain, as defined as in Equation~\ref{eq:gain}. The authors concluded that GEM combination therapies had more risk of grade 3 or 4 adverse events. This was to be expected given that NICE does not recommend GEM combination therapies for patients who may not be able to tolerate it. \\

\begin{equation}
    \frac{\frac{\text{GEM Median OS}}{HR}-\text{GEM Median OS}}{\frac{\text{GEM Median PFS}}{HR}-\text{GEM Median PFS}}
    \label{eq:gain}
\end{equation}

The Takumoto~\cite{takumoto} study considered only first line patients in Japan. In total, 25 studies were included comparing 22 treatments for OS were included in the Takumoto NMA. They found FOLFIRINOX and GEM-NAB offered improved OS compared to GEM. The reported HRs indicated that FOLFIRINOX was more efficacious than GEM-NAB. The Takumoto NMA was performed in a Bayesian framework, using an RE model. Although conducted with the Japanese clinical guidelines in mind, the included studies were two-arm studies not necessarily performed in Japan. Interestingly, the authors compared their results to the Gresham NMA, and found their results to be in line with those of Gresham. \\

The~\cite{zhang} NMA inlcuded several treatments that were also included in this NMA. They found FOLFIRINOX was the best treatment in the network. They used an RE model and summarised the results using the Surface Under the Cumulative Ranking Score (SUCRA) values. In terms of SUCRA values of 12-month OS, FOLFIRINOX performed best, followed by GEM+S1 (not included in this NMA) and GEM-NAB. 

\section{Implications of the Literature Review}
The literature showed that FOLFIRINOX and GEM-NAB are the best performing treatments for pancreatic cancer, which was to be expected given the treatment landscape. The papers consistently stated that more trials were needed, which would of course be ideal, but FOLFIRINOX and GEM-NAB are both very expensive treatments, so this is unlikely. Toxicity was considered in the majority of these NMAs, but was not considered here. \\

None of the NMAs performed any covariate adjustment, meaning there is a gap in the literature that could be filled using an ML-NMR. The issue with this is that IPD is of course required to do any meaningful covariate population adjustment. This NMA could therefore not be used to plug this gap, as only the proportion male was considered as a covariate. It was only included to facilitate using the ML-NMR method.  \\
