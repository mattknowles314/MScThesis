\chapter{Introduction}

\section{Pancreatic Cancer}
Pancreatic cancer is the $10^{th}$ most common cancer in the UK, accounting for $3\%$ of all new cancer cases~\cite{pancStat}. Pancreatic cancer has a particularly poor prognosis, with 9,558 deaths from 10,452 cases between 2016 and 2018. Only $3\%$ of patients survive for more than five years~\cite{NG85}. Part of the reason for the poor prognosis is that pancreatic cancer is hard to detect at early stages, meaning most people who present with symptoms already have advanced-stage pancreatic cancer by the time they present. Often, patients only notice symptoms when the tumour has spread to surrounding tissues, or metastises to other organs~\cite{kelsen}. The liver is the most common site of pancreatic cancer metastases~\cite{deeb}. Common symptoms of pancreatic cancer include indigestion, stomach or back pain, loss of appeteite and jauncdice~\cite{pancSymp}. These symtoms are common in other illnesses, which contributes to patients overlooking the fact their symptoms are consistent with pancreatic cancer. \\

Risk factors of pancreatic cancer include smoking, diabetes, obesity, and high-fat diets. Smoking is the dominant risk factor, with around $20\%$ of cases being caused by cigarette smoking. In addition, cancers from smokers contain more genetic mutations when compared to cancers from non-smokers~\cite{blackford}. \\

\section{Treatment Landscape}
Gemcitabine (GEM) ($C_9H_{11}F_2N_3O_4$) is a standard first-line treatment for pancreatic cancer administered intrveneously~\cite{NG85}. GEM is also used to treat other types of cancer, including breast cancer, bladder cancer and non-small-cell lung cancer~\cite{wong2009}. GEM can be administered alone or in combination with another medication.\\

This dissertation considered six treatments that were given in combination with GEM: capecitabine (CAP) ($C_{15}H_{22}FN_3O_6$), axitinib (AXI) ($C_{22}H_{18}N_4OS$), pemetrexed (PEM) ($C_{20}H_{21}N_5O_6$), sorafenib (SOR) ($C_{21}H_{16}CIF_{3}N_4O_3)$, nab-paclitaxel (NAB)\footnote{As nab-paclitaxel is a mixture of paclitaxel ($C_{47}H_{51}NO_{14}$) with albumin protein, it does not have a standard chemical formula} and irinotecan (IRI) ($C_{33}H_{38}N_4O_6$). In addition, one standalone treatment, FOLFIRINOX, was included. FOLFIRINOX is a combination of oxaliplatin ($C_{8}H_{14}N_{2}O_{4}Pt$), irinotecan ($C_{33}H_{38}N_{4}O_{6}$), leucovorin ($C_{20}H_{23}N_{7}O_{7}$), and fluorouracil ($C_{4}H_{3}FN_{2}O_{2}$), which is currently the recommended first-line treatment for metastatic pancreatic cancer in the UK.\\

A Network Meta Analysis (NMA) conducted by~\cite{gresham2014}, which included FOLFIRINOX, GEM-NAB, and GEM-CAP, found all three treatments to be associated with statisitically significant improvements in Overall Survival (OS) relative to GEM and several other treatments. Their NMA was a Bayesian NMA for calculating survival outcomes. The primary result outcomes of their NMA were the HR, and survival gain, as defined as in Equation~\ref{eq:gain}. 

\begin{equation}
    \frac{\frac{\text{GEM Median OS}}{HR}-\text{GEM Median OS}}{\frac{\text{GEM Median PFS}}{HR}-\text{GEM Median PFS}}
    \label{eq:gain}
\end{equation}

The National Insitute for Health and Care Excellence (NICE) last updated their guidance on the digangosis and management of pancreatic cancer in 2018~\cite{NG85}. The guidance for treatment is split for patients with locally advanced cancer and metastatic cancer.

\subsection{Locally Advanced Pancreatic Cancer}
Systemic combination chemotherapy is offered to patients who are well enough to tolerate. For those who are not well enough to tolerate combination therapy, GEM is offered. CAP should be considered as the radiosensitiser\footnote{A radiosensitiser is a treatment that makes cancer cells more susceptible to radiotherapy}.

\subsection{Metastatic Pancreatic Cancer}
In first line treatment, if a patient has an Eastern Cooperative Oncology Group (ECOG) performance status of $0$ to $1$ and is well enough, they are offered FOLFIRINOX. If the patient is not well enough to tolerate FOLFIRINOX, then GEM combination therapy is offered. Further, if the patient is not well enough to tolerate GEM combination therapy, then GEM monotherapy is offered. \\

In second-line treatment, oxaliplatin-based chemotherapy is offered to patients who did not recieve first-line oxaliplatin. GEM combination therapy is offered in the second-line treatment case for patients who progressed after first-line FOLFIRINOX.

\section{Project Aim}\label{sec:aims}
To assess the comparative efficacy of treatments for advanced/metastatic pancreatic cancer using Multilevel Network Meta-Regression (ML-NMR). As the focus of this project was not on a thoroguh literature review, but rather on the methodology, inclusion criteria were not particularly strict, but studies had to be a phase II or phase III trial, and contain published Kaplan-Meier (KM) curves with numbers at risk. In addition, studies had to report the proportion of male patients on each treatment arm. Only the OS endpoint was considered. \\

The secondary aim was to corroborate the findings of current guidance. NICE TA476~\cite{TA476} described some uncertainty in the comparison between GEM-NAB and GEM-CAP, so the NMA also aimed to provide some clarification on the comparative efficacy of GEM-NAB and GEM-CAP. The other piece of literature that the NMA aimed to verify was an NMA conducted in 2014~\cite{gresham2014}, which used different outcome measures. 

\section{Dissertation Structure}
Chapter~\ref{survchap} outlines some concepts in survival analysis that are important to this dissertation. The survival and hazard functions are introduced along with some key metrics. In particular, the median survival time and restricted mean survival time (RMST) are introduced. Finally, there is some discussion on parametric survival models, which the later chapters rely on. \\

Chapter~\ref{nmatheory} discusses the theory of NMAs. First, the idea of a network of evidence is outlined using graph theory. The standard NMA model is then discussed, looking at the relative effect of treatments in a network of evidence. The notion of different types of effects NMAs are introduced, and some discussion on types of variables considered in NMAs is presented. This is then expanded on to derive the ML-NMR model for general likelihoods, which is in turn expanded on to describe the ML-NMR model for survival outcomes. There is some discussion of the numerical integration techniques used when implementing the survival ML-NMR model. Following this, the outcome measures are introduced in the ML-NMR context, building on the initial definitions in Chapter~\ref{survchap}. Finally, the model selection criteria and metrics are introduced. \\

Chapter~\ref{trialschap} presents the clinical trials that were included in the NMA. A summary table presents the treatments, number of patients, and summary covariate information in each trial. There is some discussion on the similarity of the studies based on the data in the table. The discussion of similarity is aided by two forest plots presenting the variation of the median OS in each study. The KM curves for each study are also presented. Finally, the extrapolation plots of the parametric models fit to each treatment arm of each study are presented. \\

Chapter~\ref{nmachap} presents the NMA itself. First, the network of evidence is presented along with some discussion. The model fitting and selection is then presented, with discussion on which model provided the best fit and was therefore selected. A table presenting the information criterion scores for each model is used to compare each model. The results are then presented. These include survival and hazard plots, along with forest plots of median OS and RMST. \\

Chapter~\ref{conclusions} discusses the results of Chapter~\ref{nmachap} and the interpretation thereof. The aims presented in Section~\ref{sec:aims} are adressed before some more general discussion and comments on potential future work. \\

Appendix~\ref{suppdefs} contains some definitions that are referred to, but not directly required, in the outline of the analysis. Appendix~\ref{NMAAppendix} presents some additional results from the NMA that were not required for drawing any conclusions, but may be of value to the interested reader. Finally, Appendix~\ref{pack} outlines the development of the \verb|PCNMA| R package that was developed for conducting this NMA. While not relevant to understanding any of the results, this appendix may be interesting to the more programming-minded reader.