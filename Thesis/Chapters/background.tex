\chapter{Introduction}

\section{Pancreatic Cancer}
Pancreatic cancer is the $10^{th}$ most common cancer in the UK, accounting for $3\%$ of all new cases~\cite{pancStat}. Pancreatic cancer has a particularly poor prognosis. In 2016-2018, there were 9,558 deaths from 10,452 cases. Pancreatic cancer is hard to detect at early stages, meaning most people who present with symptoms already have advanced-stage pancreatic cancer. Often, patients only notice symptoms when the tumour has spread to surrounding tissues, or metastises to other organs~\cite{kelsen}. The liver is the most common site of pancreatic cancer metastases~\cite{deeb}. Common symptoms of pancreatic cancer include indigestion, stomach or back pain, loss of appeteite and jauncdice~\cite{pancSymp}. \\

Risk factors of pancreatic cancer include smoking, diabetes, obesity, and high-fat diets. Smoking is the dominant cause, with around $20\%$ of cases being caused by cigarette smoking. In addition, cancers from smokers contain more genetic mutations when compared to cancers from non-smokers~\cite{blackford}. 

\section{Treatment Landscape}
Gemcitabine (GEM) ($C_9H_{11}F_2N_3O_4$) is a standard first-line treatment for pancreatic cancer administered intrveneously~\cite{NG85}. GEM is also used to treat other types of cancer, including breast cancer, bladder cancer and non-small-cell lung cancer~\cite{wong2009}. GEM can be administered alone or in combination with another medication.\\

This dissertation considered six treatments that were given in combination with GEM: capecitabine (CAP) ($C_{15}H_{22}FN_3O_6$), axitinib (AXI) ($C_{22}H_{18}N_4OS$), pemetrexed (PEM) ($C_{20}H_{21}N_5O_6$), sorafenib (SOR) ($C_{21}H_{16}CIF_{3}N_4O_3)$, nab-paclitaxel (NAB)\footnote{As nab-paclitaxel is a mixture of paclitaxel ($C_{47}H_{51}NO_{14}$) with albumin protein, it does not have a standard chemical formula} and irinotecan (IRI) ($C_{33}H_{38}N_4O_6$). \\

A Network Meta Analysis (NMA) conducted by~\cite{gresham2014}, which included GEM, GEM-NAB, and GEM-CAP, found all three to be associated with statisitically significant improvements in Overall Survival (OS) relative to GEM, and several other treatments. Their NMA was a Bayesian NMA for calculating survival outcomes. The primary result outcomes of their NMA were the HR, and survival gain, defined as in Equation~\ref{eq:gain}. 

\begin{equation}
    \frac{\frac{\text{GEM Median OS}}{HR}-\text{GEM Median OS}}{\frac{\text{GEM Median PFS}}{HR}-\text{GEM Median PFS}}
    \label{eq:gain}
\end{equation}

\section{Project Aim}
To assess the efficacy of GEM alone versus GEM in combination with one of the six afforementioned treatments for the treatment of advanced/metastatic pancreatic cancer, a modified Multi-Level Network Meta Regression (ML-NMR) was used. As the focus of this project was not on a thoroguh literature review, but rather on the methodology, inclusion criteria were not particularly strict, but studies had to be a phase II or phase III trial, and contain published Kaplan-Meier (KM) curves with numbers at risk. In addition, studies had to report the proportion of male patients on each treatment arm. Only the OS endpoint was considered. 

