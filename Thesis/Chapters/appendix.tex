\chapter{Appendix}\label{appChap}

\section{Appendix A: Additional NMA Results}\label{NMAAppendix}

This appendix presents some additional NMA results for the FE log-normal ML-NMR model. 

\subsection{Base Case}

\begin{figure}[h]
    \centering
    \includegraphics[width = \textwidth]{../Results/NMA/Survival_Plot.png}
    \caption{OS of each treatment in each population}
    \label{fig:pred_survbc}
\end{figure}

\begin{figure}[h]
    \centering
    \includegraphics[width = \textwidth]{../Results/NMA/ranks.png}
    \caption{Rank probabilities for each treatment}
    \label{fig:rankplot}
\end{figure}

\clearpage
\subsection{Sensitivity Analysis 1}

\begin{figure}[h]
    \centering
    \includegraphics[width = \textwidth]{../Results/NMA/SA1/Survival_Plot.png}
    \caption{OS of each treatment in each population - SA1}
    \label{fig:pred_survSA1}
\end{figure}

\clearpage
\section{Sensitivity Analysis 2}

\begin{figure}[h]
    \centering
    \includegraphics[width = \textwidth]{../Results/NMA/SA2/Survival_Plot.png}
    \caption{OS of each treatment in each population - SA2}
    \label{fig:pred_survSA2}
\end{figure}

\section{Appendix B: ISPOR Good Practice Questions}\label{isporqs}

\subsection{Evidence Base}
\textbf{Is the population relevant?} Yes. The populations in these trials were reflective of people most likely to have advanced/metastatic pancreatic cancer. The only slight deviaton was the Conroy study having a maximum age eligibility requirement, meaning that study likely had fitter patients.\\ 

\textbf{Are any relevant interventions missing?} Potentially. The Gresham NMA included 19 treatments in total. This NMA included all of the best-performing treatments from the Gresham study, but could be expanded to include more studies in the future. The treatments in this NMA however were chosen as the ones most likely to be of interest to a decision maker in a clinical setting. \\ 

\textbf{Are any relevant outcomes missing?} No PFS data was included. The puprose of this NMA was for assessing the OS endpoint, but the same methods would apply with PFS data. \\

\textbf{Is the context (settings and circumstances) applicable?} Treatment for pancreatic cancer has not changed much over the last 25 years. Indeed, while the studies used in this NMA were reported between 2004 and 2015, all the treatments are still relevant in 2024. If surgical studies had been included, there may be some discussion required, as more attempts are being made to operate on pancreatic cancer nowadays. \\

\textbf{Did the researchers attempt to identify and include all
relevant RCTs?} No thorough literature was performed for this NMA. Studies were selected based on how well they were reported, due to considerations with digitising the KM curves, and whether the comparators were relevant. As mentioned previously, the best-performing treatments from the Gresham study were of primary interest for this NMA.\\

\textbf{Do the trials for the interventions of interest form one
connected network of RCTs?} Yes, see Figure~\ref{fig:osnet}.\\

\textbf{Is it apparent that poor quality studies were included, thereby
leading to bias?} No. The only potential comment here is the low sample size of the Spano or Goncalves studies. These studies were not deemed to be of poor quality despite this.\\

\textbf{Is it likely that bias was induced by selective reporting of
outcomes in the studies?} As only the OS endpoint was considered, this was not deemed to be a potential influencer of any bias in the NMA.\\

\textbf{Are there systematic differences in treatment effect modifiers
(i.e., baseline patient or study characteristics that have an
impact on the treatment effects) across the different treatment
comparisons in the network?} No. The populations in the included studies were more-or-less identical. We included the proportion of male patients as a covariate in order to use the ML-NMR method.\\

\textbf{If yes (i.e., there are such systematic differences in treatment
effect modifiers), were these imbalances in effect modifiers
across the different treatment comparisons identified before
comparing individual study results?} NA.\\

\subsection{Analysis}
\textbf{Were statistical methods used that preserve within-study
randomization? (No naive comparisons)} Yes. The NMA was based on relative treatment effects. \\

\textbf{If both direct and indirect comparisons are available for
pairwise contrasts (i.e., closed loops), was agreement in
treatment effects (i.e., consistency) evaluated or discussed?} No closed loops. Question not applicable. \\

\textbf{In the presence of consistency between direct and indirect
comparisons, were both direct and indirect evidence included in
the network meta-analysis?} No closed loops. Question not applicable. \\

\textbf{With inconsistency or an imbalance in the distribution of
treatment effect modifiers across the different types of
comparisons in the network of trials, did the researchers
attempt to minimize this bias with the analysis?} This was not deemed relevant due to the similarity of the trials. \\

\textbf{Was a valid rationale provided for the use of random-effects
or fixed-effect models?} Given the similarity of the included trials, FE models were not deemed to be clinically inappropriate. This is why both FE and RE models were fit for each likelihood. The best performing models were selected based on the LOOIC and DIC scores, rather than any clinical considerations. \\

\textbf{If a random-effects model was used, were assumptions
about heterogeneity explored or discussed?} Question not applicable as the RE model was not used.\\

\textbf{If there are indications of heterogeneity, were subgroup
analyses or meta-regression analysis with prespecified
covariates performed?} Yes, by nature of using an ML-NMR.\\

\subsection{Reporting Quality and Transparency}

\textbf{Is a graphical or tabular representation of the evidence
network provided with information on the number of RCTs per
direct comparison?} Yes, see Figure~\ref{fig:osnet}. The thickness of lines denotes the number of RCTs avaialble per comparison. \\

\textbf{Are the individual study results reported?} Yes.\\

\textbf{Are results of direct comparisons reported separately from
results of the indirect comparisons or network meta-analysis?} Question not applicable.\\

\textbf{Are all pairwise contrasts between interventions as obtained
with the network meta-analysis reported along with measures
of uncertainty?} Yes, see Figure~\ref{fig:pair_releff}.\\

\textbf{Is a ranking of interventions provided given the reported
treatment effects and its uncertainty by outcome?} Yes, see Figure~\ref{fig:sucra}. \\

\textbf{Is the effect of important patient characteristics on treatment
effects reported?} No, due to lack of IPD. \\

\subsection{Interpretation}

\textbf{Are the conclusions fair and balanced?} Yes. Every attempt was made to do this. \\

\subsection{Conflicts of Interest}

\textbf{Were there any potential conflicts of interest?} This project was somewhat personal to the author, but that personal experience did not influence the results. There was no inherent bias to a particular therapy. The author is employed by a health economics consultancy, but has no interest in a particular therapy from a commercial perspective either.\\

\textbf{If yes, were steps taken to address these?} Not required.

\section{Appendix C: The PCNMA Package}\label{pack}

To facilitiate the analysis conducted in this project, an \verb|R| package was constructed. Performing the analysis in this way ensures easy reproducibility for further analysis in the future. The package is roughly split into two parts: survival code, and NMA code. 

\subsection{Survival Functions}
The central function in the package for survival analysis is the \textit{fit\_distribution} function. This function is a wrapper around the \textit{flexsurvreg} function from \verb|flexsurv|. It is designed to take a list object containing the names of distributions and the associated argument to pass that distribution to \verb|flexsurv|. For example, 

\begin{lstlisting}
    distributions <- list("Weibull" = "weibull")
\end{lstlisting}

The reason for defining the distributions in this way is because the default \verb|flexsurv| arguments look untidy by default when plotting. For example, the exponential distribution is passed as just ``exp''. The beauty of the \textit{fit\_distribution} function is that several distributions can be passed with a tidier name, which \textit{flexsurvreg} never sees. This purely aesthetic change can be seen by comparing Figure~\ref{fig:flexsurvlabelexample}, and Figure~\ref{fig:fitdistributionexampleplot}. \\

\begin{figure}[h]
    \includegraphics{../figures/flexsurvExamplePlot.png}
    \caption{Fitted Models as in Flexsurv}
    \label{fig:flexsurvlabelexample}
\end{figure}

\begin{figure}[h]
    \includegraphics{../figures/fitteddistExamplePlot.png}
    \caption{Fitted Models with \textit{fit\_distribution}}
    \label{fig:fitdistributionexampleplot}
\end{figure}

Any default \verb|flexsurv| distribution will work with \textit{fit\_distribution}, it just needs a suitable name in the list object . The \textit{fit\_distribution} function itself doesn't actually fit any distributions. There is a sub-function, \textit{.fit\_distribution} which takes a single distribution as an argument, and some data on which to fit that distribution. This function takes the ``weibull'' element of the above list and passes it to flexsurv. The \textit{fit\_distribution} function maps \textit{.fit\_distribution} across the list of distributions, using the \verb|purrr::map| function~\cite{purrr}. After some data cleaning, the object that is returned by \textit{fit\_distribution} is given the class ``fitted\_distribution''.  \\

Several S3 methods exist for objects of class ``fitted\_distribution'', these are \textit{plot.fitted\_distribution}, \textit{summary.fitted\_distribution}, and \textit{coef.fitted\_distribution}. These functions allow for plotting fitted models, accessing information such as AIC scores, and accessing the model coefficients respectively.

\subsection{NMA Functions}
The NMA part of the \verb|PCNMA| package is a wrapper around \verb|multinma|. In particluar, the \textit{fit\_model} function is a wrapper around \textit{multinma::nma}. \\

Once a network has been created using the \verb|multinma| functions \textit{set\_ipd} and \textit{set\_agd}, the \textit{fit\_model} function, with arguments such as iterations and chains can be used to fit an ML-NMR method for a given likelihood. Models fit in this way are given the class \textit{fitted\_model}. An object of class \textit{fitted\_model} has two S3 methods: \textit{plot.fitted\_distribution}, used for plotting trace plots and posterior results, such as predicted KM curves for a given study population, and \textit{summary.fitted\_distribution}, which is used for obtaining the LOOIC and DIC scores for the model. \\

