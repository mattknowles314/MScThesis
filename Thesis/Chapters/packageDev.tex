\chapter{The PCNMA Package}\label{pack}

To facilitiate the analysis conducted in this project, an \verb|R| package was constructed. Performing the analysis in this way ensures easy reproducibility for further analysis in the future. The package is roughly split into two parts: survival code, and NMA code. 

\section{Survival Functions}
The central function in the package for survival analysis is the \textit{fit\_distribution} function. This function is a wrapper around the \textit{flexsurvreg} function from \verb|flexsurv|. It is designed to take a list object containing the names of distributions and the associated argument to pass that distribution to \verb|flexsurv|. For example, 

\begin{lstlisting}
    distributions <- list("Weibull" = "weibull")
\end{lstlisting}

The reason for defining the distributions in this way is because the default \verb|Flexsurv| arguments are somewhat untitdy. For example, the exponential distribution is passed as just ``exp''. The beauty of the \textit{fit\_distribution} function is that several distributions can be passed with a tidier name, which \textit{flexsurvreg} never sees. This purely aesthetic change can be seen by comparing Figure~\ref{fig:flexsurvlabelexample}, and Figure~\ref{fig:fitdistributionexampleplot}. \\

\begin{figure}[h]
    \includegraphics{../figures/flexsurvExamplePlot.png}
    \caption{Fitted Models as in Flexsurv}
    \label{fig:flexsurvlabelexample}
\end{figure}

\begin{figure}[h]
    \includegraphics{../figures/fitteddistExamplePlot.png}
    \caption{Fitted Models with \textit{fit\_distribution}}
    \label{fig:fitdistributionexampleplot}
\end{figure}

Any default \verb|flexsurv| distribution will work with \textit{fit\_distribution}, it just needs a suitable name in the list object . The \textit{fit\_distribution} function itself doesn't actually fit any distributions. There is a sub-function, \textit{.fit\_distribution} which takes a single distribution as an argument, and some data on which to fit that distribution. This function takes the ``weibull'' element of the above list and passes it to flexsurv. The \textit{fit\_distribution} function maps \textit{.fit\_distribution} across the list of distributions, using the \verb|purrr::map| function~\cite{purrr}. After some data cleaning, the object that is returned by \textit{fit\_distribution} is given the class ``fitted\_distribution''.  \\

Several S3 methods exist for objects of class ``fitted\_distribution'', these are \textit{plot.fitted\_distribution}, \textit{summary.fitted\_distribution}, and \textit{coef.fitted\_distribution}. These functions allow for plotting fitted models, accessing information such as AIC scores, and accessing the model coefficients respectively.