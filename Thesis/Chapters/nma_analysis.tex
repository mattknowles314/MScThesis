\chapter{NMA of Pancreatic Cancer Trials}

\section{Network of Evidence}
Figure~\ref{fig:osnet} presents the network of evidence for this NMA. Only aggregate data (AgD) was available for each study. There were two studies comparing GEM with GEM-AXI, but only one study for each other comparison. As indicated by the size of each node, GEM-SOR was the treatment with the lowest sample size, and GEM-NAB was the comparator with the highest.

\begin{figure}[h]
    \centering
    \includegraphics[scale = 0.6]{../figures/OS_Network.png}
    \caption{Network of evidence}
    \label{fig:osnet}
\end{figure}

\section{Model Description and Selection}
Both FE and RE models were fit using gamma, generlaised gamma, Gompertz, log-logistic, log-normal, and Weibull likelihoods. Vague priors were used for each model. Namely, the intercept prior was $N(0, 100)$, the treatment prior was $N(0, 10)$, the auxiliary prior was $hN(0, 5)$, and auxiliary regression prior was $N(0, 10)$. Here, $hN$ denotes a \textit{half-normal distribution}.

\begin{definition}{Half Normal Distribution}
    LLet $X$ be a normal distribution such that $X \sim N(0, \sigma^2)$. Then $Y = |X|$ follows a half-normal distribution. In particular, the half-normal distribution has PDF 
    \[
        f(x, \sigma) = \frac{\sqrt{2}}{\sigma \sqrt{\pi}}\exp \left(-\frac{x^2}{2\sigma^2}\right)    
    \]
    With $x \in [0, \infty)$, and $\sigma > 0$.
\end{definition}

For each model, sampling was done using 2000 iterations on four chains. The first 1000 iterations were warmup iterations. 