\chapter{NMA of Pancreatic Cancer Trials}\label{nmachap}

\section{Base Case}
\subsection{Network of Evidence}
Figure~\ref{fig:osnet} presents the network of evidence for this NMA. There were two studies comparing GEM with GEM-AXI, but only one study for each other comparison. As indicated by the size of each node, GEM-SOR was the treatment with the lowest sample size, and GEM-NAB was the comparator with the highest. The GEM $\to$ GEM-NAB edge is a different colour due to being an IPD trial. 

\begin{figure}[h]
    \centering
    \includegraphics[width = \textwidth]{../figures/OS_Network.png}
    \caption{Network of evidence}
    \label{fig:osnet}
\end{figure}

\subsection{Model Fitting and Selection}
Both FE and RE models were fit using log-logistic, log-normal, and Weibull likelihoods. Vague priors were used for each model. Namely, the intercept prior was $N(0, 100)$, the treatment prior was $N(0, 10)$, the auxiliary prior was $hN(0, 5)$, and auxiliary regression prior was $N(0, 10)$. Here, $hN$ denotes a \textit{half-normal distribution}. For each model, sampling was done using 1000 iterations on four chains. The first 500 iterations were warmup iterations. In total, it took approximately six days to run all the models using $64$ integration points for the numerical integration.\\

Table~\ref{tab:selectionstatbc} presents the selection statistics for each model. The RE log-logistic model performed best in terms of LOOIC and DIC score. The FE log-logistic was very similar to the RE log-logistic model. Indeed, the scores were only seperated by 0.0398. The RE log-normal model was only 0.9148 above the LOOIC score of the RE log-logistic model. Given the similarity in the LOOIC scores for the FE and RE log-logistic models, the final moel was selected based on model convergence. \\

Figure~\ref{fig:tracebcFE} and Figure~\ref{fig:tracebcRE} present the trace plots for the FE and RE log-logistic models respectively. The FE model demonstrated much better convergence than the RE model. In the RE model, the traces for each treatment were very thin with lots of spikes. In paticular, for GEM-AXI, there was a clear issue with convergence just before iteration 300. Conversely, the FE traces for each treatment showed few spikes, and the chains were well mixed, indicating good convergence. \\

Figure~\ref{fig:parcoord} presents the (standardized) parallel coordinates plot for the FE log-logistic model. Each green line represents an iteration, connecting the values of the parameter corresponding to each treatment. The value of each parameter has been standardized by the transformation given in Equation~\ref{eq:trans_parcoord}. The purpose of a parallel coordinates plot is to show divergent transitions. Figure~\ref{fig:parcoord} shows, by nature of all the iterations being green, that there were no divergent transitions. Figure~\ref{fig:RE_parcoord} shows the parallel coordinates plot for the RE log-logistic model, for which the divergent transitions are clear.\\

\begin{equation}
    f(x) = \frac{x - \mu_x}{\sigma_x}
    \label{eq:trans_parcoord}
\end{equation}

Figure~\ref{fig:pairs} and Figure~\ref{fig:RE_pairs} present the univariate histograms and bivariate scatter plots for the parameters of each treatment in the FE and RE log-logistic models respectively. The FE log-logistic histograms were much wider than there RE model counterparts. The scatter plots for each parameter were much sparser and scattered in the RE model than in the FE model. The divergent transitions are shown as red points in the scatter plot. \\

Given the clinical plausibility of the FE model and the demonstrated superior convergence compared to the RE model, the FE log-logistic model was selected.

\begin{table}[h]
    \centering
    \begin{tabular}{llll}
    \hline
    Likelihood   & Type & DIC         & LOOIC      \\ \hline
    Log-logistic & FE  & 16974.3668  & 16972.9184 \\
    Log-logistic & RE & 16972.2638  $\leftarrow$ & 16972.8786 $\leftarrow$ \\
    Log-normal   & FE  & 107813403.9532  & 48652.0393 \\
    Log-normal   & RE & 16977.7101  & 16973.7934 \\
    Weibull      & FE  & 16989.2670 & 16992.9722 \\
    Weibull      & RE & 3.1937e42 & 5.8355e21 \\ \hline
    \end{tabular}
    \caption{Model selection statistics for each model}
    \label{tab:selectionstatbc}
\end{table}

\begin{figure}[h]
    \centering
    \begin{minipage}[b]{0.45\textwidth}
        \centering
        \includegraphics[width=\textwidth]{../Results/NMA/Trace.png}
        \caption{Trace plot for the FE log-logistic model ML-NMR}
        \label{fig:tracebcFE}
    \end{minipage}
    \hspace{0.05\textwidth}
    \begin{minipage}[b]{0.45\textwidth}
        \centering
        \includegraphics[width=\textwidth]{../Results/NMA/RE_Trace.png}
        \caption{Trace plot for the RE log-logistic model ML-NMR}
        \label{fig:tracebcRE}
    \end{minipage}
\end{figure}

\begin{figure}[h]
    \centering
    \begin{minipage}[b]{0.45\textwidth}
        \centering
        \includegraphics[width = \textwidth]{../Results/NMA/Parcoord.png}
        \caption{Parallel coordinates plot of the FE log-logistic model}
        \label{fig:parcoord}
    \end{minipage}
    \hspace{0.05\textwidth}
    \begin{minipage}[b]{0.45\textwidth}
        \centering
        \includegraphics[width = \textwidth]{../Results/NMA/RE_Parcoord.png}
        \caption{Parallel coordinates plot of the RE log-logistic model}
        \label{fig:RE_parcoord}
    \end{minipage}
\end{figure}

\begin{figure}[h]
    \centering
    \includegraphics[width = \textwidth]{../Results/NMA/FE_Pairs.png}
    \caption{Pairs plot for the FE log-logistic model}
    \label{fig:pairs}
\end{figure}

\begin{figure}[h]
    \centering
    \includegraphics[width = \textwidth]{../Results/NMA/RE_Pairs.png}
    \caption{Pairs plot for the RE log-logistic model}
    \label{fig:RE_pairs}
\end{figure}


\subsection{Results}
Figure~\ref{fig:prior_post} presents a density plot of the prior versus posterior distribution for each treatment. The posterior distribution is shown in green, and the prior in red. The priors were of course uninformative priors. There were two key takeaways from Figure~\ref{fig:prior_post}. Firstly, the good convergence of this model was demonstrated by the shape of the posterior histogram through the consistent and unimodal shape of the distributions. Secondly, the shape of the posterior distributions being significantly different from the shape of the prior showed that the data were the primary influence on the parameter estimates. If the posterior were a similar shape to the prior, it would indicate the data has not contributed much to the model. Some inference about the effectiveness of these treatments compared to GEM could already be drawn from this plot. For example, FOL, GEM-NAB, and GEM-CAP are all centered above zero, indicating superior OS compared to GEM. Figure~\ref{fig:prior_post} also demonstrates the uncertainty for each interval. Indeed, GEM-SOR has a particularly wide posterior distribution, and GEM-NAB had quite a thin posterior distribution. This was to be expected as there is more uncertainty in the GEM-SOR arm due to fewer patients in the GEM-SOR arm of the network.

Figure~\ref{fig:pred_survbc_conroy} and Figure~\ref{fig:pred_survbc_goldstein} present the KM curves of each treatment in the Conroy and Goldstein populations, respectively. Figure~\ref{fig:pred_survbc}, available in Appendix~\ref{NMAAppendix} presents the KM curves in each population. As Conroy and Goldstein had the most mature data of all studies, and contained the two main treatments, these plots were given more priority in the analysis. Both Figure~\ref{fig:pred_survbc_conroy} and Figure~\ref{fig:pred_survbc_goldstein} showed superior OS for the FOL compared to all other treatments. GEM-NAB provides superior OS to GEM-CAP, but worse OS than FOL in the Conroy population. The same was observed in the Goldstein population. In both populations, GEM-SOR had the worst OS until between 12 and 15 months, but then crosses the GEM and GEM-AXI curves to give a higher OS at the end of the extrapolation period. \\

Figure~\ref{fig:pred_rmstbc} presents the estimated RMST of each treatment in each population. The credible intervals for GEM-SOR were very wide, likely due to the low number of patients in the GEM-SOR arm. The thin line for each treatment represents the 95\% credible interval. Based on this inteval, FOL was significantly better than GEM, GEM-AXI, GEM-PEM and GEM-SOR in the Conroy, Goldstein and Oettle studies. FOL was also significantly better than GEM-IRI in the Conroy, Cunningham, Goldstein, Kindler, Oettle and Rocha Lima studies. FOL was not significantly better than FOL in any study population. GEM-CAP and GEM-NAB gave similar RMST estimates in each study population. Indeed, GEM-NAB provided a higher RMST estiamte in each population, but this improvement was nigt significant. In addition, GEM, GEM-AXI and GEM-PEM gave similar estimates in each treatment arm, with GEM-PEM providing slightly better RMST then GEM and GEM-AXI in each case. GEM-IRI gave higher RMST estimates than GEM-SOR in each population, but worse RMST estimates than GEM, GEM-AXI, and GEM-PEM. \\

Figure~\ref{fig:pred_medianbc} presents the estimated median OS of each treatment in each population. The median OS estimates followes the same pattern as the RMST estimates. Namely, GEM-SOR and GEM-IRI gave the lowest and second-lowest estimates for median OS in each study population, GEM, GEM-AXI, and GEM-PEM gave similar estimates, and FOL and GEM-NAB gave the highest and second-highest estimates of median OS, respectively. Compared to the RMST estimates, the credible interval for FOL was quite word in the Goldstein, Goncalves, Kindler, and Spano studies. GEM-NAB provides higher median OS estimates than GEM-CAP in each popultion, but does not provide significant improvements. The similarity between GEM, GEM-AXI, and GEM-PEM that was observed in the RMST plots was also observed in the median OS plots. \\ 

Figure~\ref{fig:releff} presents the population-average relative treatment effects of each treatment in terms of the log survival-time from the median OS estimates. FOL and GEM-NAB were both significantly better than GEM, but all other treatments crossed 0, indicating no significance. GEM-CAP did not provide a significant imporovement to OS comapred to GEM. In addition, Figure~\ref{fig:pair_releff} presents the relative treatment effects for all constrats in the network. As expected, all studies except GEM-CAP and GEM-NAB are significantly worse than FOL.  \\

Figure~\ref{fig:sucra} presents the cumulative rank probability for each treatment. FOL had a considerably higher probability of being the best treatment. FOL had a probability of being the best treatment of 0.97. GEM-NAB and and GEM-CAP had probabilities of being the best treatment of 0.02 and 0.01 respectively. The cumulative probability of GEM-NAB increased quicker than GEM-CAP, having a cumulative probability of being the second best treatment of 0.66, compared to GEM-CAP having a cumulative probability of $0.28$. FOL achies a cumulative probability of 1 by rank three, and GEM-NAB and GEM-CAP achieved a cumulative probability of 1 at rank five and seven resepectively. Figure~\ref{fig:rankplot}, available in Appendix~\ref{NMAAppendix} presents the non-cumulative posterior ranks. The SUCRA value of FOL, GEM-NAB, and GEM-CAP was $99.5\%$, $80.3\%$, and $71.0\%$ respectively. The SUCRA values of GEM-SOR and GEM-IRI were $15.8\%$ and $20.7\%$, respectively which were the lowest SUCRA values in the netork. GEM-AXI, GEM-PEM, and GEM had SUCRA values of $37.8\%$, $39.0\%$, and $36.0\%$, respectively. 

\begin{figure}[h]
    \centering
    \includegraphics[width = \textwidth]{../Results/NMA/prior_post.png}
    \caption{Prior versus posterior distribution for each treatment}
    \label{fig:prior_post}
\end{figure}

\begin{figure}[h]
    \centering
    \includegraphics[width = \textwidth]{../Results/NMA/Conroy_Survival_Plot.png}
    \caption{OS of each treatment in the Conroy population}
    \label{fig:pred_survbc_conroy}
\end{figure}

\begin{figure}[h]
    \centering
    \includegraphics[width = \textwidth]{../Results/NMA/Goldstein_Survival_Plot.png}
    \caption{OS of each treatment in the Goldstein population}
    \label{fig:pred_survbc_goldstein}
\end{figure}

\begin{figure}[h]
    \centering
    \includegraphics[width = \textwidth]{../Results/NMA/RMST_Plot.png}
    \caption{RMST of each treatment in each population}
    \label{fig:pred_rmstbc}
\end{figure}

\begin{figure}[h]
    \centering
    \includegraphics[width = \textwidth]{../Results/NMA/Median_Plot.png}
    \caption{Median OS of each treatment in each population}
    \label{fig:pred_medianbc}
\end{figure}

\begin{figure}[h]
    \centering
    \includegraphics[width = \textwidth]{../Results/NMA/Releff.png}
    \caption{Relative treatment effects for all treatments versus GEM}
    \label{fig:releff}
\end{figure}

\begin{figure}[h]
    \centering
    \includegraphics[width = \textwidth]{../Results/NMA/Pair_Releff.png}
    \caption{Pairwise relative treatment effects for all treatments}
    \label{fig:pair_releff}
\end{figure}

\begin{figure}[h]
    \centering
    \includegraphics[width = \textwidth]{../Results/NMA/SUCRA.png}
    \caption{Cumulative rank probability for each treatment}
    \label{fig:sucra}
\end{figure}

\clearpage
\section{Sensitivity Analyses}

\subsection{Sensitivity Analysis 1}
The first SA was used to account for the fact that the Conroy study had a stricter age eligibility criteria than the other studies. By removing FOL, this SA also accounted for the fact that FOL can only be given to patients fit enough to take it, and therefore decision makers may not always have FOL as a treatment option. Figure~\ref{fig:sa1net} presents the network of evidence for this SA. The only difference between this and the base case evidence network was the absence of a GEM $\rightarrow$ FOL comparison. \\

\subsubsection{Model Fitting and Selection}
Table~\ref{selectStatSA1} presents the model selection statistics. The RE log-normal and FE-log normal scored best in terms of LOOIC and DIC score, respectively. As in the base case, both models were assessed to determine which one provided a better fit. Figure~\ref{fig:traceSA1FE} and Figure~\ref{fig:traceSA1RE} present the trace plots for the FE and RE model respectively. The FE model had better traces for each treatment parameter than the RE model. In addition, Figure~\ref{fig:parcoord_SA1} and Figure~\ref{fig:RE_parcoord_SA1} present the normalised parallel coordinates plot for the FE and RE model, respectively. The FE model showed no divergent transitions, with intermingling chains. The RE model showed many divergent transitions. In particular, several chains had a completely different pattern to the remainder, indicating that the sampling may have gotten stuck in a local minima or maxima. The RE model was therefore not deemed to have achieved acceptable convergence, and the FE log-normal model was used. The pairs plot for the FE and RE model are presented in Figure~\ref{fig:pairs_SA1} and Figure~\ref{fig:RE_pairs_SA1}. \\

\begin{figure}[h]
    \centering
    \includegraphics[width = \textwidth]{../figures/SA1_Network.png}
    \caption{Network of evidence}
    \label{fig:sa1net}
\end{figure}

\begin{table}[h]
    \centering
    \begin{tabular}{lllll}
    \hline
    Likelihood   & Effect & DIC        & LOOIC         \\ \hline
    Log-logistic & Fixed  & 15176.1347 & 15177.3745    \\
    Log-logistic & Random & 15178.0459 & 15177.7154    \\
    Log-normal   & Fixed  & 15172.7599 $\leftarrow$ & 15174.3034    \\
    Log-normal   & Random & 15172.9396 & 15173.1961 $\leftarrow$   \\
    Weibull      & Fixed  & 15203.4555 & 15203.4465    \\
    Weibull      & Random & Inf        & 23994604.6763 \\ \hline
    \end{tabular}
    \caption{Model selection statistics for each model}
    \label{selectStatSA1}
\end{table}

\begin{figure}[h]
    \centering
    \begin{minipage}[b]{0.45\textwidth}
        \centering
        \includegraphics[width=\textwidth]{../Results/NMA/SA1/Trace.png}
        \caption{Trace plot for the FE log-normal model ML-NMR}
        \label{fig:traceSA1FE}
    \end{minipage}
    \hspace{0.05\textwidth}
    \begin{minipage}[b]{0.45\textwidth}
        \centering
        \includegraphics[width=\textwidth]{../Results/NMA/SA1/RE_Trace.png}
        \caption{Trace plot for the RE log-normal model ML-NMR}
        \label{fig:traceSA1RE}
    \end{minipage}
\end{figure}

\begin{figure}[h]
    \centering
    \begin{minipage}[b]{0.45\textwidth}
        \centering
        \includegraphics[width = \textwidth]{../Results/NMA/SA1/Parcoord.png}
        \caption{Parallel coordinates plot of the FE log-normal model}
        \label{fig:parcoord_SA1}
    \end{minipage}
    \hspace{0.05\textwidth}
    \begin{minipage}[b]{0.45\textwidth}
        \centering
        \includegraphics[width = \textwidth]{../Results/NMA/SA1/RE_Parcoord.png}
        \caption{Parallel coordinates plot of the RE log-normal model}
        \label{fig:RE_parcoord_SA1}
    \end{minipage}
\end{figure}

\begin{figure}[h]
    \centering
    \includegraphics[width = \textwidth]{../Results/NMA/SA1/FE_Pairs.png}
    \caption{Pairs plot for the FE log-normal model}
    \label{fig:pairs_SA1}
\end{figure}

\begin{figure}[h]
    \centering
    \includegraphics[width = \textwidth]{../Results/NMA/SA1/RE_Pairs.png}
    \caption{Pairs plot for the RE log-normal model}
    \label{fig:RE_pairs_SA1}
\end{figure}

\subsubsection{Results}
Figure~\ref{fig:prior_post_SA1} presents the prior versus posterior distribution for each treatment. The posterior distributions were significantly different from the priors, indicating the data was primary influence on the parameter estimates. The GEM-SOR parameter had a wider posterior than the other treatments due to less available information for the GEM-SOR treatment arm. \\

Figure~\ref{fig:pred_surv_goldstein_SA1} presents the simulated KM curves of each treatment in the Goldstein population. GEM-SOR was the worst treatment in this population until around 25 monhts, where it becomes comparable to GEM. GEM-NAB had the highest OS throughout the entire observation period. GEM-CAB and GEM-NAB had similar OS until around five months, after which the GEM-CAP curve declines quicker than GEM-NAB. \\

Figure~\ref{fig:pred_rmst_SA1} presents the RMST estimates for each treatment in each population. GEM-NAB had the highest RMST estimate in each population. GEM-CAP had the second highest RMST of each treatment in each population. In the Goncalves and Kindler studies, the RMST estimates of GEM-CAP and GEM-NAB were quite similar. GEM, GEM-AXI, and GEM-PEM had very similar RMST estimates in each population. GEM-SOR and GEM-IRI had the lowest and second-lowest RMST estimates in each population. Figure~\ref{fig:pred_median_SA1} presents the median OS estimates of each treatment in each population. The same pattern was observed in the median OS estimates as in the RMST estimates. The gap between GEM-NAB and GEM-CAP was slightly more pronounced in the median OS estimates than in the RMST estimates. \\

Figure~\ref{fig:releff_SA1} presents the relative effects of each treatment versus GEM in terms of the log survival time ratio. GEM-AXI, GEM-PEM, GEM-CAP, and GEM-NAB were all better than GEM. Only GEM-NAB was significantly better than GEM. Figure~\ref{fig:pair_releff_SA1} presents the relative effects for all comparisons. This plot showed that GEM-NAB was better than GEM-CAP, but not significantly, which was consitent with the base case analysis. GEM-NAB was significantly better than GEM-IRI. \\

Figure~\ref{fig:sucra_SA1} presents the cumulative rank probability plot. GEM-NAB and GEM-CAP had the highest probabilities of being the best treatment. The SUCRA values of GEM-NAB and GEM-CAP were $60.8\%$ and $30.0\%$ respectively. GEM-SOR had a higher probability of being the best, second best, third best, and fourth best treatment than GEM-IRI, and a higher probability of being the best and sedond best treatment than GEM. GEM-PEM and GEM-AXI had similar probabilities for each ranking.

\begin{figure}[h]
    \centering
    \includegraphics[width = \textwidth]{../Results/NMA/SA1/prior_post.png}
    \caption{Prior versus posterior distribution for each treatment}
    \label{fig:prior_post_SA1}
\end{figure}

\begin{figure}[h]
    \centering
    \includegraphics[width = \textwidth]{../Results/NMA/SA1/Goldstein_Survival_Plot.png}
    \caption{OS of each treatment in the Goldstein population}
    \label{fig:pred_surv_goldstein_SA1}
\end{figure}

\begin{figure}[h]
    \centering
    \includegraphics[width = \textwidth]{../Results/NMA/SA1/RMST_Plot.png}
    \caption{RMST of each treatment in each population}
    \label{fig:pred_rmst_SA1}
\end{figure}

\begin{figure}[h]
    \centering
    \includegraphics[width = \textwidth]{../Results/NMA/SA1/Median_Plot.png}
    \caption{Median OS of each treatment in each population}
    \label{fig:pred_median_SA1}
\end{figure}

\begin{figure}[h]
    \centering
    \includegraphics[width = \textwidth]{../Results/NMA/SA1/Releff.png}
    \caption{Relative treatment effects for all treatments versus GEM}
    \label{fig:releff_SA1}
\end{figure}

\begin{figure}[h]
    \centering
    \includegraphics[width = \textwidth]{../Results/NMA/SA1/Pair_Releff.png}
    \caption{Pairwise relative treatment effects for all treatments}
    \label{fig:pair_releff_SA1}
\end{figure}

\begin{figure}[h]
    \centering
    \includegraphics[width = \textwidth]{../Results/NMA/SA1/SUCRA.png}
    \caption{Cumulative rank probability for each treatment}
    \label{fig:sucra_SA1}
\end{figure}

\clearpage
\subsection{Sensitivity Analysis 2}
The second SA excluded considered trials for which the OS in both treatment arms dropped below $0.25$. By including studies with more mature data, this SA aimed to assess any uncertainty in the base case by removing these trials with immature data. The Goncalves and Kindler studies were therefore removed. 

\subsubsection{Model fitting and selection}
Figure~\ref{fig:sa2net} presents the network of evidence for this SA. The differences between this network and the base case were the loss of one GEM $\rightarrow$ GEM-AXI comparison and the loss of the GEM $\rightarrow$ GEM-SOR comparison. Given that GEM-SOR was the worst ranked treatment in the base case, dropping this comparison was not deemed to be an issue. The RE log-logistic model scored best in terms of both LOOIC and DIC scores Figure~\ref{fig:traceSA2FE} and Figure~\ref{fig:traceSA2RE} present the trace plots for the FE and RE model respectively. The FE model had better traces for each treatment parameter than the RE model. In addition, Figure~\ref{fig:parcoord_SA1} and Figure~\ref{fig:RE_parcoord_SA1} present the normalised parallel coordinates plot for the FE and RE model, respectively. Unlike the base case and sensitivity analysis 1, the RE model did not have any divergent transitions. In order to determine which model was used, the pairs plots were examined. Figure~\ref{fig:pairs_SA2} and Figure~\ref{fig:RE_pairs_SA2} Given present the pairs plots for the FE and RE models respectively. The shapes of the distributions in the RE model were thinner than the FE model. The FE model was therefore selected as the better fitting model. \\

\begin{figure}[h]
    \centering
    \includegraphics[width = \textwidth]{../figures/SA2_Network.png}
    \caption{Network of evidence}
    \label{fig:sa2net}
\end{figure}

\begin{table}[h]
    \centering
    \begin{tabular}{llll}
    \hline
    Likelihood   & Effect & DIC        & LOOIC      \\ \hline
    Log-logistic & Fixed  & 14970.7099 & 14968.6730 \\
    Log-logistic & Random & 14970.2523 $\leftarrow$ & 14968.1193 $\leftarrow$ \\
    Log-normal   & Fixed  & 14971.6948 & 14973.3886 \\
    Log-normal   & Random & 14972.6324 & 14973.0334 \\
    Weibull      & Fixed  & 14992.7850 & 14996.2477 \\
    Weibull      & Random & 14992.2891 & 14995.4664 \\ \hline
    \end{tabular}
    \caption{Model selection statistics for each model}
    \label{selectStatSA2}
\end{table}

\begin{figure}[h]
    \centering
    \begin{minipage}[b]{0.45\textwidth}
        \centering
        \includegraphics[width=\textwidth]{../Results/NMA/SA2/Trace.png}
        \caption{Trace plot for the FE log-normal model ML-NMR}
        \label{fig:traceSA2FE}
    \end{minipage}
    \hspace{0.05\textwidth}
    \begin{minipage}[b]{0.45\textwidth}
        \centering
        \includegraphics[width=\textwidth]{../Results/NMA/SA2/RE_Trace.png}
        \caption{Trace plot for the RE log-normal model ML-NMR}
        \label{fig:traceSA2RE}
    \end{minipage}
\end{figure}

\begin{figure}[h]
    \centering
    \begin{minipage}[b]{0.45\textwidth}
        \centering
        \includegraphics[width = \textwidth]{../Results/NMA/SA2/Parcoord.png}
        \caption{Parallel coordinates plot of the FE log-normal model}
        \label{fig:parcoord_SA2}
    \end{minipage}
    \hspace{0.05\textwidth}
    \begin{minipage}[b]{0.45\textwidth}
        \centering
        \includegraphics[width = \textwidth]{../Results/NMA/SA2/RE_Parcoord.png}
        \caption{Parallel coordinates plot of the RE log-normal model}
        \label{fig:RE_parcoord_SA2}
    \end{minipage}
\end{figure}

\begin{figure}[h]
    \centering
    \includegraphics[width = \textwidth]{../Results/NMA/SA2/FE_Pairs.png}
    \caption{Pairs plot for the FE log-normal model}
    \label{fig:pairs_SA2}
\end{figure}

\begin{figure}[h]
    \centering
    \includegraphics[width = \textwidth]{../Results/NMA/SA2/RE_Pairs.png}
    \caption{Pairs plot for the RE log-normal model}
    \label{fig:RE_pairs_SA2}
\end{figure}

\subsubsection{Results}
Figure~\ref{fig:prior_post_SA2} presents the prior versus posterior distribution for each treatment. The posterior distributions were significantly different from the priors, indicating the data was primary influence on the parameter estimates. The GEM-AXI parameter had a wider posterior than the other treatments. \\

Figure~\ref{fig:pred_surv_goldstein_SA2} and Figure~\ref{fig:pred_surv_conroy_SA2} presents the simulated KM curves of each treatment in the Goldstein and Conroy population respectively. FOL had the highest OS in both populations. GEM-NAB had the second highest OS in both tretment arms. GEM-AXI had a higher OS than GEM-NAB until just before ten months. GEM-IRI had the lowest OS in the Goldstein population until just after ten months, when GEM and GEM-PEM became the worst performing treatment. GEM and GEM-PEM had similar OS curves in the Goldstein population. GEM-IRI had the lowest OS in the Conroy population until around ten months, when GEM and GEM-PEM again become the worst performing treatments. \\

Figure~\ref{fig:pred_rmst_SA2} presents the RMST estimates for each treatment in each population. The RMST estimates were more spread out in each population than in the base case and first sensitivity analysis results. FOL had the highest RMST in each population. GEM-NAB had the second highest RMST estimate in each population. GEM-CAP did not have RMST estimates as close to GEM-NAB as in the base case and first sensitivity analysis. GEM-AXI had higher RMST estimates than GEM-CAP in each population, but was noted for having very wide credible intervals. GEM-IRI had the lowest RMST estimate in each population. GEM and GEM-PEM had similar RMST estimates in each population. Figure~\ref{fig:pred_median_SA2} presents the median OS estimates of each treatment in each population. GEM-AXI had similar median OS estimates to GEM-NAB, but did not have higher median OS than GEM-NAB in any treatment population. FOL had the highest median OS in each population. GEM-IRI had the lowest median OS estimates in each population. \\

Figure~\ref{fig:releff_SA2} presents the relative effects of each treatment versus GEM in terms of the log survival time ratio. FOL, GEM-AXI, GEM-CAP, GEM-NAB, and GEM-PEM were all better than GEM. Only FOL and GEM-NAB was significantly better than GEM. GEM-AXI had a higher log survival time ratio than GEM-CAP, and a very similar estimate to GEM-NAB. Figure~\ref{fig:pair_releff_SA2} presents the relative effects for all comparisons. This plot showed that GEM-NAB was better than GEM-CAP, but not significantly, which was consitent with the base case analysis. GEM-NAB was not significantly better than GEM-IRI, as it was in the first sensitivity analysis. GEM-IRI and GEM-PEM were significantly worse than FOL. GEM-NAB and GEM-AXI were very similar. \\

Figure~\ref{fig:sucra_SA2} presents the cumulative rank probability plot. FOL and GEM-AXI had the highest probabilities of being the best treatment. FOL and GEM-AXI had SUCRA values of $84.2\%$ and $14.3\%$, respectively. GEM-NAB did not have a higher SUCRA value than GEM-CAP until after the second best treatment. GEM-IRI had the lowest probability of being the fourth to seventh treatment. 

\begin{figure}[h]
    \centering
    \includegraphics[width = \textwidth]{../Results/NMA/SA2/prior_post.png}
    \caption{Prior versus posterior distribution for each treatment}
    \label{fig:prior_post_SA2}
\end{figure}

\begin{figure}[h]
    \centering
    \includegraphics[width = \textwidth]{../Results/NMA/SA2/Goldstein_Survival_Plot.png}
    \caption{OS of each treatment in the Goldstein population}
    \label{fig:pred_surv_goldstein_SA2}
\end{figure}

\begin{figure}[h]
    \centering
    \includegraphics[width = \textwidth]{../Results/NMA/SA2/Conroy_Survival_Plot.png}
    \caption{OS of each treatment in the Conroy population}
    \label{fig:pred_surv_conroy_SA2}
\end{figure}

\begin{figure}[h]
    \centering
    \includegraphics[width = \textwidth]{../Results/NMA/SA2/RMST_Plot.png}
    \caption{RMST of each treatment in each population}
    \label{fig:pred_rmst_SA2}
\end{figure}

\begin{figure}[h]
    \centering
    \includegraphics[width = \textwidth]{../Results/NMA/SA2/Median_Plot.png}
    \caption{Median OS of each treatment in each population}
    \label{fig:pred_median_SA2}
\end{figure}

\begin{figure}[h]
    \centering
    \includegraphics[width = \textwidth]{../Results/NMA/SA2/Releff.png}
    \caption{Relative treatment effects for all treatments versus GEM}
    \label{fig:releff_SA2}
\end{figure}

\begin{figure}[h]
    \centering
    \includegraphics[width = \textwidth]{../Results/NMA/SA2/Pair_Releff.png}
    \caption{Pairwise relative treatment effects for all treatments}
    \label{fig:pair_releff_SA2}
\end{figure}

\begin{figure}[h]
    \centering
    \includegraphics[width = \textwidth]{../Results/NMA/SA2/SUCRA.png}
    \caption{Cumulative rank probability for each treatment}
    \label{fig:sucra_SA2}
\end{figure}
