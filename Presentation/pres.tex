\documentclass{beamer}

\usetheme{Rochester}
\usecolortheme{whale}

\AtBeginSection[]{
  \begin{frame}
  \vfill
  \centering
  \begin{beamercolorbox}[sep=8pt,center,shadow=true,rounded=true]{title}
    \usebeamerfont{title}\insertsectionhead\par%
  \end{beamercolorbox}
  \vfill
  \end{frame}
}


\title{A Network Meta-Analysis of Treatments for Locally Advanced/Metastatic Pancreatic Cancer}
\author{Matthew Knowles}
\institute{University of Sheffield}
\date{August 12 2024}

\begin{document}

\begin{frame}
    \titlepage
\end{frame} 

\section{Background}

\begin{frame}{Pancreatic Cancer}
    \begin{itemize}
        \item Pancreatic cancer is the $10^{th}$ most common cancer in the UK, accounting for roughly $3\%$ of all new cases. 
        \item The disease is associated with a particular poor prognosis, primarily due to late diagnoses.
        \item Most cases are pancreatic duct  adenocarcinomas (PDAC), which form in the exocrine component of the pancreas. This part of the organ is responsible for producing digestive enzymes, and carrying the enzymes away from the pancreas. 
    \end{itemize}
\end{frame}

\begin{frame}{Treatment Landscape}
    \begin{itemize}
        \item Gemcitabine (GEM) is a standard, not-particularly toxic treatment for pancreatic cancer.
        \item GEM in combination with capecitabine (GEM-CAP) or GEM in combination with nab-Paclitaxel (GEM-NAB) have been shown to be better than GEM alone.
        \item FOLFIRINOX (FOL) has been shown to be significantly better than GEM, but is only given to patients who can tolerate it, which is not many. 
        \item In current NICE guidance, there is uncertainty around the comparison between GEM-CAP and GEM-NAB.
        \item Several NMAs have been conducted on pancreatic cancer trials, but none using the Multilevel Network Meta-Regression (ML-NMR) framework.
    \end{itemize} 
\end{frame}

\begin{frame}{Project outline}
    \begin{enumerate}
        \item Digitise published Overall Survival (OS) Kaplan-Meier (KM) curves from some pancreatic cancer trials.
        \pause
        \item Fit parametric survival models, and select a few best-fitting model candidates.
        \pause
        \item Use those models as likelihoods in the ML-NMR.
        \pause
        \item Conduct NMA, assess best treatments.
    \end{enumerate} 
\end{frame} 

\begin{frame}{Project aims}
    \begin{enumerate}
        \item Determine the best treatment for pancreatic cancer using an ML-NMR.
        \pause
        \item Provide clarity on the comparison between GEM-NAB and GEM-CAP.
        \pause
        \item Corroborate findings of previous NMAs.
    \end{enumerate} 
\end{frame} 

\section{Methodology}

\begin{frame}{Survival Analysis}
    \begin{itemize}
        \item By looking at the KM curves for each study, the Proportional Hazards Assumption was deemed to be violated.
        \pause
        \item Therefore, the standard parametric models (excluding the exponential model) were fit to the data from each study.
        \pause
        \item Model fitting was conducted using ther R package ``flexsurv''.
        \pause
        \item Based on the Akaike's Information Criterion (AIC) scores, the log-logistic, log-normal, and Weibull models were selected. 
    \end{itemize}   
\end{frame} 

\begin{frame}{ML-NMR Background}
    \begin{itemize}
        \item ML-NMR is a nice method for performing an NMA with a mixture of individual patient data (IPD) and aggregate (AgD) data. 
        \pause
        \item The ML-NMR method lends itself really well to survival outcomes.
    \end{itemize}
\end{frame}

\begin{frame}{General ML-NMR Model}
    \begin{block}{Definition: ML-NMR for general likelihoods}
    Individual:
    \begin{align}
        L_{ijk|x}^{\text{Con}}(\xi;y_{ijk},x_{ijk}) &= \pi_{\text{Ind}}(y_{ijk}|\theta_{ijk}) \\
        g(\theta_{ijk}) &= \eta_{jk}(x_{ijk}) = \mu_j + x_{ijk}^T(\beta_1 + \beta_{2,k}) + \gamma_k \label{mlnmragg}
    \end{align}
    Aggregate:
    \begin{align}
        L_{ijk}^{\text{Mar}}(\xi; y_{ijk}) &= \int_{\mathfrak{X}} L_{ijk|x}^{\text{Con}}(\xi; y_{ijk}, x)f_{jk}(x)dx \label{mlnnmrint}\\
        L_{\hat{jk}}^{\text{Mar}} &\propto \prod_{i = 1}^{N_{jk}}L_{ijk}^{\text{Mar}}(\xi; y_{ijk})
    \end{align}
    \end{block}
\end{frame}

\begin{frame}{Survival ML-NMR}
    \begin{itemize}
        \item Consider now that you have some survival data $y_{ijk} = \{t_{ijk}, c_{ijk}\}$. 
        \pause
        \item In the individual case, individual $i$ also has a vector of covariate values $x_{ijk}$. For the aggregate data, the only covariate information available is the distribution at baseline $f_{jk}$. 
        \pause 
        \item Let $S_{jk}(t|x)$ and $h_{jk}(t|x)$ be the survival and hazard functions at time $t$ conditional on the covariates $x$. 
        \pause
    \item The conditional likelihood contributions for each individual at time $t_{ijk}$ is given by
        \[L_{ijk|x}^{Con}(\zeta;t_{ijk},c_{ijk},x_{ijk}) = S_{jk}(t_{ijk}| x_{ijk})h_{jk}(t_{ijk}|x_{ijk})^{c_{ijk}}\]
        \pause 
        \item \textbf{The form of the survival and hazard function depends on the survival model chosen}.
    \end{itemize}
\end{frame}

\begin{frame}{Survival ML-NMR}
    \begin{itemize}
        \item We now want to derive the marginal likelihood.
        \pause
        \item \[L_{ijk}^{\text{Mar}}(\xi; y_{ijk}) = \int_{\mathfrak{X}} L_{ijk|x}^{\text{Con}}(\xi; y_{ijk}, x)f_{jk}(x)dx\]
        \pause
        \item Substitute $y_{ijk} = \{t_{ijk}, c_{ijk}\}$
        \pause 
        \item \[L_{ijk}^{\text{Mar}}(\xi; t_{ijk}, c_{ijk}) = \int_{\mathfrak{X}} L_{ijk|x}^{\text{Con}}(\xi; t_{ijk}, c_{ijk}, x)f_{jk}(x)dx\]
                                          \[ = \int_{\mathfrak{X}} S_{jk}(t_{ijk}| x_{ijk})h_{jk}(t_{ijk}|x_{ijk})^{c_{ijk}} f_{jk}(x)dx\]
    \end{itemize}
\end{frame}

\section{Trials}

\begin{frame}{KM Curves}
    \begin{figure}
        \includegraphics[height=0.9\textheight,keepaspectratio]{../figures/OS_KMs.png}
    \end{figure}    
\end{frame}

\begin{frame}{KM Considerations}
    \begin{itemize}
        \item The Goncalves and Kindler studies had comparatively immature data. They were included in the main NMA, but a sensitivity analysis was conducted where they were excluded. 
        \pause
        \item In the Kindler, Oettle, and Rocha Lima studies, the GEM and comparator arm were very similar. 
        \pause
        \item Conroy and Goldstein have data on FOL and GEM-NAB respectively. The apparent improvement in OS that was observed in the literature review is visible from these studies. 
    \end{itemize}
\end{frame}

\begin{frame}{Network of Evidence}
    \begin{figure}
        \includegraphics[height=0.9\textheight,keepaspectratio]{../figures/OS_network.png}
    \end{figure}    
\end{frame}

\section{Results}

\begin{frame}{Model Selection Statistics}
\begin{table}
    \centering
    \begin{tabular}{llll}
    \hline
    Likelihood   & Type & DIC         & LOOIC      \\ \hline
    Log-logistic & FE  & 16974.3668  & 16972.9184 \\
    Log-logistic & RE & 16972.2638  & 16972.8786 \\
    Log-normal   & FE  & 107813403.9532  & 48652.0393 \\
    Log-normal   & RE & 16977.7101  & 16973.7934 \\
    Weibull      & FE  & 16989.2670 & 16992.9722 \\
    Weibull      & RE & 3.1937e42 & 5.8355e21 \\ \hline
    \end{tabular}
\end{table}
\end{frame}

\begin{frame}{Model Selection Statistics}
\begin{table}
    \centering
    \begin{tabular}{llll}
    \hline
    Likelihood   & Type & DIC         & LOOIC      \\ \hline
    Log-logistic & FE  & 16974.3668  & 16972.9184 \\
    Log-logistic & RE & 16972.2638  $\leftarrow$ & 16972.8786 $\leftarrow$ \\
    Log-normal   & FE  & 107813403.9532  & 48652.0393 \\
    Log-normal   & RE & 16977.7101  & 16973.7934 \\
    Weibull      & FE  & 16989.2670 & 16992.9722 \\
    Weibull      & RE & 3.1937e42 & 5.8355e21 \\ \hline
    \end{tabular}
\end{table}
\end{frame}

\begin{frame}{Model Selection Considerations}
    \begin{itemize}
        \item The RE log-logistic model performed best, but notice how the FE model was quite close.
        \pause
        \item FE models were deemed to be clinically appropriate, and so it was decided to assess the fit of both models, rather than selecting purely on DIC scores.
        \pause
        \item The model fit was assessed using trace plots. 
    \end{itemize}
\end{frame}

\begin{frame}{Assessing Model Fit}
\begin{figure}
    \centering
    \begin{minipage}[b]{0.45\textwidth}
        \centering
        \includegraphics[width=\textwidth]{../Results/NMA/Trace.png}
    \end{minipage}
    \hspace{0.05\textwidth}
    \begin{minipage}[b]{0.45\textwidth}
        \centering
        \includegraphics[width=\textwidth]{../Results/NMA/RE_Trace.png}
    \end{minipage}
\end{figure}

The left figure is the FE model, and the right figure is the RE model. Clearly, the FE model had better convergence
\end{frame}   

\begin{frame}{Predicted KM Curves}
\begin{figure}
    \centering
    \includegraphics[width = 0.825\textwidth]{../Results/NMA/Survival_Plot.png}
\end{figure}
\end{frame}

\begin{frame}{Predicted Median OS}
\begin{figure}
    \centering
    \includegraphics[width = 0.825\textwidth]{../Results/NMA/Median_Plot.png}
\end{figure}
\end{frame}

\begin{frame}{Predicted Restricted Mean Survival Time}
\begin{figure}
    \centering
    \includegraphics[width = 0.825\textwidth]{../Results/NMA/RMST_Plot.png}
\end{figure}
\end{frame}

\begin{frame}{Relative effectiveness compared to GEM}
\begin{figure}
    \centering
    \includegraphics[width = 0.825\textwidth]{../Results/NMA/Releff.png}
\end{figure}
\end{frame}

\section{Conclusions and Future Work}
\begin{frame}{Conclusions}
    \begin{itemize}
        \item FOLFIRINOX provided superior OS than GEM-NAB and GEM-CAP. (Aim 1) 
        \pause
        \item GEM-NAB provides slightly better OS than GEM-CAP, but not significantly better. (Aim 2)
        \pause
        \item GEM-SOR and GEM-IRI provided worse OS than GEM. 
    \end{itemize}
\end{frame}

\begin{frame}{Discussion}
    \begin{itemize}
        \item These results were in line with several NMAs that included the same comparators (Aim 3).
        \pause
        \item The Conroy study capped the upper age of patients at 76. A sensitivity analysis was performed where the FOL study was excluded. In this NMA, the treatment order remained the same as in the main NMA. The comparison between GEM-NAB and GEM-CAP still showed similar OS.
        \pause
        \item The ISPOR good practice guidelines for NMAs were considered to ensure validity of the NMA. These were addressed in an appendix of the dissertation itself.
        \pause
        \item Future work will include more studies. It would be nice to implement some form of parallel-processing, to make the computing process more efficient.
    \end{itemize}
\end{frame}

\section{Thank you}

\end{document}
