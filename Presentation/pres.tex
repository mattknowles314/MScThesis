\documentclass{beamer}

\usetheme{Rochester}
\usecolortheme{whale}

\AtBeginSection[]{
  \begin{frame}
  \vfill
  \centering
  \begin{beamercolorbox}[sep=8pt,center,shadow=true,rounded=true]{title}
    \usebeamerfont{title}\insertsectionhead\par%
  \end{beamercolorbox}
  \vfill
  \end{frame}
}


\title{A Network Meta-Analysis for Treatments of Locally Advanced/Metastatic Pancreatic Cancer}
\author{Matthew Knowles}
\institute{University of Sheffield}
\date{Autumn 2024}

\begin{document}

\begin{frame}
    \titlepage
\end{frame} 

\section{Background}

\begin{frame}{Pancreatic Cancer}
    \begin{itemize}
        \item Pancreatic cancer is the $10^{th}$ most common cancer in the UK, accounting for roughly $3\%$ of all new cases. 
        \item The disease is associated with a particular poor prognosis, primarily due to late diagnoses.
        \item Post cases are pancreatic duct  adenocarcinomas (PDAC), which form in the exocrine component of the pancreas. This part of the organ is responsible for producing digestive enzymes, and carrying the enzymes away from the pancreas. 
    \end{itemize}
\end{frame}

\begin{frame}{Treatment Landscape}
    \begin{itemize}
        \item Gemcitabine (GEM) is a standard, not-particularly toxic treatment for pancreatic cancer
        \item GEM in comination with capcitabine (GEM-CAP) or GEM in combination with nab-Paclitaxel (GEM-NAB) have been shown to be better than GEM alone.
        \item FOLFIRINOX (FOL) has been shown to be significantly better than GEM, but is only given to patients who can tolerate it, which is not many. 
        \item IN current NICE guidance, there is uncertainty around the comparison between GEM-CAP and GEM NAB.
        \item Several NMAs have been conducted on pancreatic cancer trials, but none using the Multilevel Network Meta-Regression framework.
    \end{itemize} 
\end{frame}

\begin{frame}{Project outline}
    \begin{enumerate}
        \item Digitise published Kaplan-Meier (KM) curves from some pancreatic cancer trials
        \pause
        \item Fit parametric survival models, and select a few best-fitting model candidates.
        \pause
        \item Use those models as likelihoods in the ML-NMR
        \pause
        \item Assess treatment landscape
    \end{enumerate} 
\end{frame} 

\begin{frame}{Project aims}
    \begin{enumerate}
        \item Determine the best treatment for pancreatic cancer using an ML-NMR
        \pause
        \item Provide clarity on the comparison between GEM-NAB and GEM-CAP
        \pause
        \item Corroborate findings of previous NMAs
    \end{enumerate} 
\end{frame} 

\section{Methodology}

\begin{frame}{Survival Analysis}
    \begin{itemize}
        \item By looking at the KM curves for each study, the Proportional Hazards Assumption was deemed to be violated.
        \pause
        \item Therefore, the standard parametric models (excluding the exponential model) were fit to the data from each study.
        \pause
        \item Model fitting was conducted using ther R package ``flexsurv''.
        \pause
        \item Based on the Akaike's Information Criterion (AIC) scores, the log-logistic, log-normal, and Weibull models were selected. 
    \end{itemize}   
\end{frame} 

\begin{frame}{ML-NMR Background}
    \begin{itemize}
        \item 
    \end{itemize}
\end{frame}

\begin{frame}{General ML-NMR Model}
    \begin{block}{Definition: ML-NMR for general likelihoods}
    Individual:
    \begin{align}
        L_{ijk|x}^{\text{Con}}(\xi;y_{ijk},x_{ijk}) &= \pi_{\text{Ind}}(y_{ijk}|\theta_{ijk}) \\
        g(\theta_{ijk}) &= \eta_{jk}(x_{ijk}) = \mu_j + x_{ijk}^T(\beta_1 + \beta_{2,k}) + \gamma_k \label{mlnmragg}
    \end{align}
    Aggregate:
    \begin{align}
        L_{ijk}^{\text{Mar}}(\xi; y_{ijk}) &= \int_{\mathfrak{X}} L_{ijk|x}^{\text{Con}}(\xi; y_{ijk}, x)f_{jk}(x)dx \label{mlnnmrint}\\
        L_{\hat{jk}}^{\text{Mar}} &\propto \prod_{i = 1}^{N_{jk}}L_{ijk}^{\text{Mar}}(\xi; y_{ijk})
    \end{align}
    \end{block}
\end{frame}

\begin{frame}{Survival ML-NMR}
    \begin{itemize}
        \item Consider now that you have some survival data $y_{ijk} = \{t_{ijk}, c_{ijk}\}$. 
        \pause
        \item In the individual case, individual $i$ also has a vector of covariate values $x_{ijk}$. For the aggregate data, the only covariate information available is the distribution at baseline $f_{jk}$. 
        \pause 
        \item Let $S_{jk}(t|x)$ and $h_{jk}(t|x)$ be the survival and hazard functions at time $t$ conditional on the covariates $x$. 
        \pause
        \item The conditional likelihood contributions for each individual is given by
        \[L_{ijk|x}^{Con}(\zeta;t_{ijk},c_{ijk},x_{ijk}) = S_{jk}(t_{ijk}| x_{ijk})h_{jk}(t_{ijk}|x_{ijk})^{c_{ijk}}\]
        \pause 
        \item \textbf{The form of the survival and hazard function depends on the survival model chosen}.
    \end{itemize}
\end{frame}

\begin{frame}{Survival ML-NMR}
    \begin{itemize}
        \item We now want to derive the marginal likelihood.
        \pause
        \item \[L_{ijk}^{\text{Mar}}(\xi; y_{ijk}) = \int_{\mathfrak{X}} L_{ijk|x}^{\text{Con}}(\xi; y_{ijk}, x)f_{jk}(x)dx\]
        \pause
        \item Substitute $y_{ijk} = \{t_{ijk}, c_{ijk}\}$
        \pause 
        \item \[L_{ijk}^{\text{Mar}}(\xi; t_{ijk}, c_{ijk}) = \int_{\mathfrak{X}} L_{ijk|x}^{\text{Con}}(\xi; t_{ijk}, c_{ijk}, x)f_{jk}(x)dx\]
                                          \[ = \int_{\mathfrak{X}} S_{jk}(t_{ijk}| x_{ijk})h_{jk}(t_{ijk}|x_{ijk})^{c_{ijk}} f_{jk}(x)dx\]
    \end{itemize}
\end{frame}

\section{Trials}

\begin{frame}{KM Curves}
    \begin{figure}
        \includegraphics[height=0.9\textheight,keepaspectratio]{../figures/OS_KMs.png}
    \end{figure}    
\end{frame}

\begin{frame}{KM Considerations}
    \begin{itemize}
        \item The Goncalves and Kindler studies had comparatively immature data. 
        \pause
        \item In the Kindler, Oettle, and Rocha Lima studies, the GEM and comparator arm were very similar. 
        \pause
        \item Conroy and Goldstein have data on FOL and GEM-NAB respectively. The apparent improvement in OS that was observed in the literature review is visible from these studies. 
        \pause
        \item The PHA was not assumed for this study, 
    \end{itemize}
\end{frame}

\section{ML-NMR}
\begin{frame}{Network of Evidence}
    \begin{figure}
        \includegraphics[height=0.9\textheight,keepaspectratio]{../figures/OS_network.png}
    \end{figure}    
\end{frame}

\begin{frame}{Model Selection Statistics}
\begin{table}
    \centering
    \begin{tabular}{llll}
    \hline
    Likelihood   & Type & DIC         & LOOIC      \\ \hline
    Log-logistic & FE  & 16974.3668  & 16972.9184 \\
    Log-logistic & RE & 16972.2638  & 16972.8786 \\
    Log-normal   & FE  & 107813403.9532  & 48652.0393 \\
    Log-normal   & RE & 16977.7101  & 16973.7934 \\
    Weibull      & FE  & 16989.2670 & 16992.9722 \\
    Weibull      & RE & 3.1937e42 & 5.8355e21 \\ \hline
    \end{tabular}
\end{table}
\end{frame}

\begin{frame}{Model Selection Statistics}
\begin{table}
    \centering
    \begin{tabular}{llll}
    \hline
    Likelihood   & Type & DIC         & LOOIC      \\ \hline
    Log-logistic & FE  & 16974.3668  & 16972.9184 \\
    Log-logistic & RE & 16972.2638  $\leftarrow$ & 16972.8786 $\leftarrow$ \\
    Log-normal   & FE  & 107813403.9532  & 48652.0393 \\
    Log-normal   & RE & 16977.7101  & 16973.7934 \\
    Weibull      & FE  & 16989.2670 & 16992.9722 \\
    Weibull      & RE & 3.1937e42 & 5.8355e21 \\ \hline
    \end{tabular}
\end{table}
\end{frame}

\begin{frame}{Model Selection Considerations}
    \begin{itemize}
        \item The RE log-logistic model performed best, but notice how the FE model was quite close.
        \pause
        \item FE models were deemed to be clinically appropriate, and so it was decided to assess the fit of both models, rather than selecting purely on DIC scores.
        \pause
        \item We assessed fit with trace plots
    \end{itemize}
\end{frame}

\begin{frame}{Assessing Model Fit}
\begin{figure}
    \centering
    \begin{minipage}[b]{0.45\textwidth}
        \centering
        \includegraphics[width=\textwidth]{../Results/NMA/Trace.png}
    \end{minipage}
    \hspace{0.05\textwidth}
    \begin{minipage}[b]{0.45\textwidth}
        \centering
        \includegraphics[width=\textwidth]{../Results/NMA/RE_Trace.png}
    \end{minipage}
\end{figure}

The left figure is the FE model, and the right figure is the RE model. Clearly, the FE model had better convergence
\end{frame}   

\begin{frame}{Predicted KM Curves}

\begin{figure}
    \centering
    \includegraphics[width = 0.85\textwidth]{../Results/NMA/Survival_Plot.png}
\end{figure}
    
\end{frame}




\end{document}
